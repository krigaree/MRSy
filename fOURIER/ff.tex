\documentclass[]{article}
\usepackage{lmodern}
\usepackage{amssymb,amsmath}
\usepackage{ifxetex,ifluatex}
\usepackage{fixltx2e} % provides \textsubscript
\ifnum 0\ifxetex 1\fi\ifluatex 1\fi=0 % if pdftex
  \usepackage[T1]{fontenc}
  \usepackage[utf8]{inputenc}
\else % if luatex or xelatex
  \ifxetex
    \usepackage{mathspec}
  \else
    \usepackage{fontspec}
  \fi
  \defaultfontfeatures{Ligatures=TeX,Scale=MatchLowercase}
\fi
% use upquote if available, for straight quotes in verbatim environments
\IfFileExists{upquote.sty}{\usepackage{upquote}}{}
% use microtype if available
\IfFileExists{microtype.sty}{%
\usepackage{microtype}
\UseMicrotypeSet[protrusion]{basicmath} % disable protrusion for tt fonts
}{}
\usepackage[margin=1in]{geometry}
\usepackage{hyperref}
\hypersetup{unicode=true,
            pdftitle={fourier},
            pdfauthor={Me},
            pdfborder={0 0 0},
            breaklinks=true}
\urlstyle{same}  % don't use monospace font for urls
\usepackage{color}
\usepackage{fancyvrb}
\newcommand{\VerbBar}{|}
\newcommand{\VERB}{\Verb[commandchars=\\\{\}]}
\DefineVerbatimEnvironment{Highlighting}{Verbatim}{commandchars=\\\{\}}
% Add ',fontsize=\small' for more characters per line
\usepackage{framed}
\definecolor{shadecolor}{RGB}{248,248,248}
\newenvironment{Shaded}{\begin{snugshade}}{\end{snugshade}}
\newcommand{\KeywordTok}[1]{\textcolor[rgb]{0.13,0.29,0.53}{\textbf{{#1}}}}
\newcommand{\DataTypeTok}[1]{\textcolor[rgb]{0.13,0.29,0.53}{{#1}}}
\newcommand{\DecValTok}[1]{\textcolor[rgb]{0.00,0.00,0.81}{{#1}}}
\newcommand{\BaseNTok}[1]{\textcolor[rgb]{0.00,0.00,0.81}{{#1}}}
\newcommand{\FloatTok}[1]{\textcolor[rgb]{0.00,0.00,0.81}{{#1}}}
\newcommand{\ConstantTok}[1]{\textcolor[rgb]{0.00,0.00,0.00}{{#1}}}
\newcommand{\CharTok}[1]{\textcolor[rgb]{0.31,0.60,0.02}{{#1}}}
\newcommand{\SpecialCharTok}[1]{\textcolor[rgb]{0.00,0.00,0.00}{{#1}}}
\newcommand{\StringTok}[1]{\textcolor[rgb]{0.31,0.60,0.02}{{#1}}}
\newcommand{\VerbatimStringTok}[1]{\textcolor[rgb]{0.31,0.60,0.02}{{#1}}}
\newcommand{\SpecialStringTok}[1]{\textcolor[rgb]{0.31,0.60,0.02}{{#1}}}
\newcommand{\ImportTok}[1]{{#1}}
\newcommand{\CommentTok}[1]{\textcolor[rgb]{0.56,0.35,0.01}{\textit{{#1}}}}
\newcommand{\DocumentationTok}[1]{\textcolor[rgb]{0.56,0.35,0.01}{\textbf{\textit{{#1}}}}}
\newcommand{\AnnotationTok}[1]{\textcolor[rgb]{0.56,0.35,0.01}{\textbf{\textit{{#1}}}}}
\newcommand{\CommentVarTok}[1]{\textcolor[rgb]{0.56,0.35,0.01}{\textbf{\textit{{#1}}}}}
\newcommand{\OtherTok}[1]{\textcolor[rgb]{0.56,0.35,0.01}{{#1}}}
\newcommand{\FunctionTok}[1]{\textcolor[rgb]{0.00,0.00,0.00}{{#1}}}
\newcommand{\VariableTok}[1]{\textcolor[rgb]{0.00,0.00,0.00}{{#1}}}
\newcommand{\ControlFlowTok}[1]{\textcolor[rgb]{0.13,0.29,0.53}{\textbf{{#1}}}}
\newcommand{\OperatorTok}[1]{\textcolor[rgb]{0.81,0.36,0.00}{\textbf{{#1}}}}
\newcommand{\BuiltInTok}[1]{{#1}}
\newcommand{\ExtensionTok}[1]{{#1}}
\newcommand{\PreprocessorTok}[1]{\textcolor[rgb]{0.56,0.35,0.01}{\textit{{#1}}}}
\newcommand{\AttributeTok}[1]{\textcolor[rgb]{0.77,0.63,0.00}{{#1}}}
\newcommand{\RegionMarkerTok}[1]{{#1}}
\newcommand{\InformationTok}[1]{\textcolor[rgb]{0.56,0.35,0.01}{\textbf{\textit{{#1}}}}}
\newcommand{\WarningTok}[1]{\textcolor[rgb]{0.56,0.35,0.01}{\textbf{\textit{{#1}}}}}
\newcommand{\AlertTok}[1]{\textcolor[rgb]{0.94,0.16,0.16}{{#1}}}
\newcommand{\ErrorTok}[1]{\textcolor[rgb]{0.64,0.00,0.00}{\textbf{{#1}}}}
\newcommand{\NormalTok}[1]{{#1}}
\usepackage{graphicx,grffile}
\makeatletter
\def\maxwidth{\ifdim\Gin@nat@width>\linewidth\linewidth\else\Gin@nat@width\fi}
\def\maxheight{\ifdim\Gin@nat@height>\textheight\textheight\else\Gin@nat@height\fi}
\makeatother
% Scale images if necessary, so that they will not overflow the page
% margins by default, and it is still possible to overwrite the defaults
% using explicit options in \includegraphics[width, height, ...]{}
\setkeys{Gin}{width=\maxwidth,height=\maxheight,keepaspectratio}
\IfFileExists{parskip.sty}{%
\usepackage{parskip}
}{% else
\setlength{\parindent}{0pt}
\setlength{\parskip}{6pt plus 2pt minus 1pt}
}
\setlength{\emergencystretch}{3em}  % prevent overfull lines
\providecommand{\tightlist}{%
  \setlength{\itemsep}{0pt}\setlength{\parskip}{0pt}}
\setcounter{secnumdepth}{0}
% Redefines (sub)paragraphs to behave more like sections
\ifx\paragraph\undefined\else
\let\oldparagraph\paragraph
\renewcommand{\paragraph}[1]{\oldparagraph{#1}\mbox{}}
\fi
\ifx\subparagraph\undefined\else
\let\oldsubparagraph\subparagraph
\renewcommand{\subparagraph}[1]{\oldsubparagraph{#1}\mbox{}}
\fi

%%% Use protect on footnotes to avoid problems with footnotes in titles
\let\rmarkdownfootnote\footnote%
\def\footnote{\protect\rmarkdownfootnote}

%%% Change title format to be more compact
\usepackage{titling}

% Create subtitle command for use in maketitle
\newcommand{\subtitle}[1]{
  \posttitle{
    \begin{center}\large#1\end{center}
    }
}

\setlength{\droptitle}{-2em}

  \title{fourier}
    \pretitle{\vspace{\droptitle}\centering\huge}
  \posttitle{\par}
    \author{Me}
    \preauthor{\centering\large\emph}
  \postauthor{\par}
      \predate{\centering\large\emph}
  \postdate{\par}
    \date{5 listopada 2018}


\begin{document}
\maketitle

\begin{Shaded}
\begin{Highlighting}[]
\ImportTok{import} \NormalTok{numpy }\ImportTok{as} \NormalTok{np}
\ImportTok{import} \NormalTok{matplotlib.pyplot }\ImportTok{as} \NormalTok{plt}
\KeywordTok{def} \NormalTok{series(x):}
    \NormalTok{const }\OperatorTok{=} \DecValTok{2} \OperatorTok{*} \NormalTok{np.pi }\OperatorTok{*} \NormalTok{x }\OperatorTok{/} \NormalTok{T}
    \ControlFlowTok{return} \KeywordTok{lambda} \NormalTok{x: An(n) }\OperatorTok{*} \NormalTok{np.cos(const }\OperatorTok{*} \NormalTok{n) }\OperatorTok{+} \NormalTok{Bn(n) }\OperatorTok{*} \NormalTok{np.sin(const }\OperatorTok{*} \NormalTok{n)}
\KeywordTok{def} \NormalTok{fab1():}
    \ControlFlowTok{return} \KeywordTok{lambda} \NormalTok{n: }\DecValTok{0}\NormalTok{, }\KeywordTok{lambda} \NormalTok{n: (}\OperatorTok{-}\DecValTok{2}\OperatorTok{*}\NormalTok{(}\OperatorTok{-}\DecValTok{1}\NormalTok{)}\OperatorTok{**}\NormalTok{n)}\OperatorTok{/}\NormalTok{(np.pi }\OperatorTok{*} \NormalTok{n)}
\KeywordTok{def} \NormalTok{fab2():}
    \ControlFlowTok{return} \KeywordTok{lambda} \NormalTok{n: }\DecValTok{1}\OperatorTok{/}\NormalTok{(np.pi}\OperatorTok{**}\DecValTok{2} \OperatorTok{*} \NormalTok{n}\OperatorTok{**}\DecValTok{2}\NormalTok{), }\KeywordTok{lambda} \NormalTok{n: }\OperatorTok{-}\DecValTok{1}\OperatorTok{/}\NormalTok{(np.pi }\OperatorTok{*} \NormalTok{n)}
\KeywordTok{def} \NormalTok{fab3():                                                              }
    \ControlFlowTok{return} \KeywordTok{lambda} \NormalTok{n: }\DecValTok{0}\NormalTok{, }\KeywordTok{lambda} \NormalTok{n: }\DecValTok{1} \ControlFlowTok{if} \NormalTok{n}\OperatorTok{==}\DecValTok{1} \ControlFlowTok{else} \DecValTok{0}   
\KeywordTok{def} \NormalTok{fun():}
    \ControlFlowTok{return} \KeywordTok{lambda} \NormalTok{x: x, }\KeywordTok{lambda} \NormalTok{x: x}\OperatorTok{**}\DecValTok{2}\NormalTok{, }\KeywordTok{lambda} \NormalTok{x: np.sin(x)}
\KeywordTok{def} \NormalTok{plotSeries(x, S, fx):}
    \NormalTok{fig }\OperatorTok{=} \NormalTok{plt.figure()}
    \NormalTok{ax }\OperatorTok{=} \NormalTok{fig.add_subplot(}\DecValTok{111}\NormalTok{)}
    \NormalTok{ax.plot(x, S, x, fx(x), }\StringTok{'r--'}\NormalTok{)}
    \NormalTok{plt.show()}
    
\CommentTok{# rng = int(input('Wpisz range: '))}
\NormalTok{rng }\OperatorTok{=} \DecValTok{100}
\NormalTok{A }\OperatorTok{=} \NormalTok{[}\OperatorTok{-}\DecValTok{1}\NormalTok{, }\DecValTok{0}\NormalTok{, }\OperatorTok{-}\NormalTok{np.pi]}
\NormalTok{B }\OperatorTok{=} \NormalTok{[}\DecValTok{1}\NormalTok{, }\DecValTok{1}\NormalTok{, np.pi]}
\NormalTok{A0 }\OperatorTok{=} \NormalTok{[}\DecValTok{0}\NormalTok{, }\DecValTok{2}\OperatorTok{/}\DecValTok{3}\NormalTok{, }\DecValTok{0}\NormalTok{] }
\NormalTok{F }\OperatorTok{=} \NormalTok{\{}\StringTok{"x"} \NormalTok{: [fab1()], }\StringTok{"x^2"} \NormalTok{: [fab2()], }\StringTok{"sin(x)"} \NormalTok{: [fab3()]\}}
\NormalTok{fx }\OperatorTok{=} \NormalTok{[fun()]}
\NormalTok{X }\OperatorTok{=} \NormalTok{[]}
\NormalTok{Tt }\OperatorTok{=} \NormalTok{[]}
\ControlFlowTok{for} \NormalTok{i }\OperatorTok{in} \BuiltInTok{range}\NormalTok{(}\DecValTok{3}\NormalTok{):}
    \NormalTok{X.append(np.linspace(A[i], B[i], num}\OperatorTok{=}\DecValTok{1000}\NormalTok{))}
    \NormalTok{Tt.append(B[i] }\OperatorTok{-} \NormalTok{A[i])}
\NormalTok{X }\OperatorTok{=} \NormalTok{np.array(X)}
\NormalTok{Tt }\OperatorTok{=} \NormalTok{np.array(Tt)}
\ControlFlowTok{for} \NormalTok{inx, (k, v) }\OperatorTok{in} \BuiltInTok{enumerate}\NormalTok{(F.items()):}
    \NormalTok{a }\OperatorTok{=} \NormalTok{A[inx] }
    \NormalTok{b }\OperatorTok{=} \NormalTok{B[inx]}
    \NormalTok{a0 }\OperatorTok{=} \NormalTok{A0[inx]}
    \NormalTok{T }\OperatorTok{=} \NormalTok{Tt[inx] }
    \NormalTok{x }\OperatorTok{=} \NormalTok{X[inx]}
    \NormalTok{An, Bn }\OperatorTok{=} \NormalTok{v[}\DecValTok{0}\NormalTok{][}\DecValTok{0}\NormalTok{], v[}\DecValTok{0}\NormalTok{][}\DecValTok{1}\NormalTok{]}
    \NormalTok{f }\OperatorTok{=} \NormalTok{series(x)}
    \NormalTok{S }\OperatorTok{=} \NormalTok{np.zeros((x.shape))}
    \ControlFlowTok{for} \NormalTok{i }\OperatorTok{in} \BuiltInTok{range}\NormalTok{(}\DecValTok{1}\NormalTok{, rng}\DecValTok{+1}\NormalTok{):}
        \NormalTok{n }\OperatorTok{=} \NormalTok{i}
        \NormalTok{S }\OperatorTok{+=} \NormalTok{f(x)}
    \NormalTok{S }\OperatorTok{+=} \NormalTok{a0}\OperatorTok{/}\DecValTok{2}
    
    \NormalTok{plotSeries(x,S, fx[}\DecValTok{0}\NormalTok{][inx])}
\end{Highlighting}
\end{Shaded}

\includegraphics{ff_files/figure-latex/unnamed-chunk-1-1.pdf}
\includegraphics{ff_files/figure-latex/unnamed-chunk-1-2.pdf}
\includegraphics{ff_files/figure-latex/unnamed-chunk-1-3.pdf}


\end{document}
