\section{Ogólne zagadnienia brzegowe II rzędu}
\subsection{Warunki brzegowe typu Dirichletta}

Rozważając najbardziej ogólne zagadnienie brzegowe rzędu II z warunkami brzegowymi typu Dirichletta postaci:

\[
\begin{cases}
\vspace{0.1cm} 
\hspace{0,1cm} a(x) u'' + b(x)u' + c(x)u =f(x) \\
\vspace{0.1cm}
\hspace{0,1cm}u|_{x=a} = u_{a} \\
\hspace{0,1cm}u|_{x=b} = u_{b}
\end{cases}
\]
, gdzie:
$x\in[a,b]$
\\
\\
Okazuje się że stosowanie schematów na pochodną różnego rzędu o tej samej dokładności skutkuje tym, że cała metoda jest również rzędu II.

\subsubsection{Cel ćwiczenia}
Na laboratorium otrzymaliśmy następujące równania dla których mieliśmy stworzyć rozwiązujący je algorytm.

a)
\[
\begin{cases}
\vspace{0.1cm} 
\hspace{0,1cm} u'' - 4u =-4x \\
\vspace{0.1cm}
\hspace{0,1cm}u|_{x=0} = 0 \\
\hspace{0,1cm}u|_{x=1} = 2
\end{cases}
\]
, gdzie:
$x\in[0,1]$
\\
Rozwiązanie analityczne: $\widetilde{u}(x) = e^2(e^4-1)^{-1} (e^{2x} - e^{-2x}) + x$
\\
\\
b)
\[
\begin{cases}
\vspace{0.1cm} 
\hspace{0,1cm} u'' -u'- 2u =cos(x) \\
\vspace{0.1cm}
\hspace{0,1cm}u|_{x=0} = -\dfrac{3}{10} \\
\hspace{0,1cm}u|_{x=\dfrac{\pi}{2}} = -\dfrac{1}{10}
\end{cases}
\]
, gdzie:
$x\in[0, \dfrac{\pi}{2}]$
\\
Rozwiązanie analityczne: $\widetilde{u}(x) = -\dfrac{1}{10}(sin(x) + 3cos(x))$
\\
\\
c)
\[
\begin{cases}
\vspace{0.1cm} 
\hspace{0,1cm} -x^2u'' +2xu' +2u=-4x^2 \\
\vspace{0.1cm}
\hspace{0,1cm}u|_{x=0} = 0 \\
\hspace{0,1cm}u|_{x=1} = 0
\end{cases}
\]
, gdzie:
$x\in[0,1]$
\\
Rozwiązanie analityczne: $\widetilde{u}(x) = x^2 - x$

\vspace{0.3cm}
Ponadto zaprezentujemy wykres porównujący rozwiązanie numeryczne z rozwiązaniem analitycznym, a także wykres błędu $||E||_{\infty}$ w zależności od liczby obranych węzłów (n).
%\begin{samepage}
\newpage
\subsubsection{Rozwiązanie}
%\textbf{Warunek brzegowy Dirichleta (I rodzaju)}
Poniżej przedstawiony został algorytm rozwiązania powyższych równań z przykładowymi danymi wejściowymi z podpunktu a).


\begin{samepage}
    \input{Lab4/zad1}
\end{samepage}

\newpage
\subsubsection{Wykresy}

a)\\
\begin{samepage}
Dla 10 węzłów:

%{\centering
    
 %   \includegraphics{Lab4/charts/zad1/zad1_n_10.png}
    
%}
\FloatBarrier
\begin{figure}[!ht]
    \begin{center}
        \includegraphics[width=0.8\textwidth]{Lab4/charts/zad1/zad1_n_10.png}
    \end{center}
    %\caption{Schemat logiczny w edytorze LAD oraz tabela symboli}
    %\label{fig:picture}
\end{figure}
\FloatBarrier
\end{samepage}

\begin{samepage}
Dla 25 węzłów:
\begin{figure}[!ht]
    \begin{center}
\includegraphics[width=0.8\textwidth]{Lab4/charts/zad1/zad1_n_25.png}
    \end{center}
    %\caption{Schemat logiczny w edytorze LAD oraz tabela symboli}
    %\label{fig:picture}
\end{figure}
\FloatBarrier
\end{samepage}

\newpage
\begin{samepage}
    
Dla 100 węzłów:
\begin{figure}[!ht]
    \begin{center}
    \includegraphics[width=0.8\textwidth]{Lab4/charts/zad1/zad1_n_100.png}
    \end{center}
    %\caption{Schemat logiczny w edytorze LAD oraz tabela symboli}
    %\label{fig:picture}
\end{figure}
\FloatBarrier
\end{samepage}
    

    
\begin{samepage}

Dla 500 węzłów:

\begin{figure}[!ht]
    \begin{center}
\includegraphics[width=0.8\textwidth]{Lab4/charts/zad1/zad1_n_500.png}
    \end{center}
    %\caption{Schemat logiczny w edytorze LAD oraz tabela symboli}
    %\label{fig:picture}
\end{figure}
\FloatBarrier
\end{samepage}

\newpage
\begin{samepage}
Dla 750 węzłów:

\begin{figure}[!ht]
    \begin{center}
    \includegraphics[width=0.8\textwidth]{Lab4/charts/zad1/zad1_n_750.png}
    \end{center}
    %\caption{Schemat logiczny w edytorze LAD oraz tabela symboli}
    %\label{fig:picture}
\end{figure}
\FloatBarrier
\end{samepage}
    

\begin{samepage}
Dla 1000 węzłów:

\begin{figure}[!ht]
    \begin{center}
    \includegraphics[width=0.8\textwidth]{Lab4/charts/zad1/zad1_n_1000.png}
    \end{center}
    %\caption{Schemat logiczny w edytorze LAD oraz tabela symboli}
    %\label{fig:picture}
\end{figure}
\FloatBarrier
\end{samepage}    

\newpage

\begin{samepage}
Błąd metody w zależności od liczby węzłów:
    \begin{figure}[!ht]
        \begin{center}
\includegraphics[width=0.8\textwidth]{Lab4/charts/zad1/error_dane.png}
        \end{center}
        %\caption{Schemat logiczny w edytorze LAD oraz tabela symboli}
        %\label{fig:picture}
    \end{figure}
    \FloatBarrier
\end{samepage} 

\begin{samepage}
    
    \begin{figure}[!ht]
        \begin{center}
\includegraphics[width=0.8\textwidth]{Lab4/charts/zad1/error.png}
        \end{center}
        %\caption{Schemat logiczny w edytorze LAD oraz tabela symboli}
        %\label{fig:picture}
    \end{figure}
    \FloatBarrier
\end{samepage} 


\newpage
b)\\
\begin{samepage}
    Dla 10 węzłów:
    
    %{\centering
    
    %   \includegraphics{Lab4/charts/zad1/zad1_n_10.png}
    
    %}
    \FloatBarrier
    \begin{figure}[!ht]
        \begin{center}
            \includegraphics[width=0.8\textwidth]{Lab4/charts/zad2/zad2_n_10.png}
        \end{center}
        %\caption{Schemat logiczny w edytorze LAD oraz tabela symboli}
        %\label{fig:picture}
    \end{figure}
    \FloatBarrier
\end{samepage}

\begin{samepage}
    Dla 25 węzłów:
    \begin{figure}[!ht]
        \begin{center}
            \includegraphics[width=0.8\textwidth]{Lab4/charts/zad2/zad2_n_25.png}
        \end{center}
        %\caption{Schemat logiczny w edytorze LAD oraz tabela symboli}
        %\label{fig:picture}
    \end{figure}
    \FloatBarrier
\end{samepage}

\newpage
\begin{samepage}
    
    Dla 100 węzłów:
    \begin{figure}[!ht]
        \begin{center}
            \includegraphics[width=0.8\textwidth]{Lab4/charts/zad2/zad2_n_100.png}
        \end{center}
        %\caption{Schemat logiczny w edytorze LAD oraz tabela symboli}
        %\label{fig:picture}
    \end{figure}
    \FloatBarrier
\end{samepage}


\begin{samepage}
    Dla 1000 węzłów:
    
    \begin{figure}[!ht]
        \begin{center}
            \includegraphics[width=0.8\textwidth]{Lab4/charts/zad2/zad2_n_1000.png}
        \end{center}
        %\caption{Schemat logiczny w edytorze LAD oraz tabela symboli}
        %\label{fig:picture}
    \end{figure}
    \FloatBarrier
\end{samepage}    

\newpage

\begin{samepage}
    Błąd metody w zależności od liczby węzłów:
    \begin{figure}[!ht]
        \begin{center}
            \includegraphics[width=0.8\textwidth]{Lab4/charts/zad2/error_dane.png}
        \end{center}
        %\caption{Schemat logiczny w edytorze LAD oraz tabela symboli}
        %\label{fig:picture}
    \end{figure}
    \FloatBarrier
\end{samepage} 

\begin{samepage}
    
    \begin{figure}[!ht]
        \begin{center}
            \includegraphics[width=0.8\textwidth]{Lab4/charts/zad2/error.png}
        \end{center}
        %\caption{Schemat logiczny w edytorze LAD oraz tabela symboli}
        %\label{fig:picture}
    \end{figure}
    \FloatBarrier
\end{samepage}   

\newpage
c)\\
\begin{samepage}
    Dla 10 węzłów:
    
    %{\centering
    
    %   \includegraphics{Lab4/charts/zad1/zad1_n_10.png}
    
    %}
    \FloatBarrier
    \begin{figure}[!ht]
        \begin{center}
            \includegraphics[width=0.8\textwidth]{Lab4/charts/zad3/zad3_n_10.png}
        \end{center}
        %\caption{Schemat logiczny w edytorze LAD oraz tabela symboli}
        %\label{fig:picture}
    \end{figure}
    \FloatBarrier
\end{samepage}

\begin{samepage}
    Dla 25 węzłów:
    \begin{figure}[!ht]
        \begin{center}
            \includegraphics[width=0.8\textwidth]{Lab4/charts/zad3/zad3_n_25.png}
        \end{center}
        %\caption{Schemat logiczny w edytorze LAD oraz tabela symboli}
        %\label{fig:picture}
    \end{figure}
    \FloatBarrier
\end{samepage}

\newpage
\begin{samepage}
    
    Dla 100 węzłów:
    \begin{figure}[!ht]
        \begin{center}
            \includegraphics[width=0.8\textwidth]{Lab4/charts/zad3/zad3_n_100.png}
        \end{center}
        %\caption{Schemat logiczny w edytorze LAD oraz tabela symboli}
        %\label{fig:picture}
    \end{figure}
    \FloatBarrier
\end{samepage}


\begin{samepage}
    Dla 1000 węzłów:
    
    \begin{figure}[!ht]
        \begin{center}
            \includegraphics[width=0.8\textwidth]{Lab4/charts/zad3/zad3_n_1000.png}
        \end{center}
        %\caption{Schemat logiczny w edytorze LAD oraz tabela symboli}
        %\label{fig:picture}
    \end{figure}
    \FloatBarrier
\end{samepage}    

\newpage

\begin{samepage}
    Błąd metody w zależności od liczby węzłów:
    \begin{figure}[!ht]
        \begin{center}
            \includegraphics[width=0.8\textwidth]{Lab4/charts/zad3/error_dane.png}
        \end{center}
        %\caption{Schemat logiczny w edytorze LAD oraz tabela symboli}
        %\label{fig:picture}
    \end{figure}
    \FloatBarrier
\end{samepage} 

\begin{samepage}
    
    \begin{figure}[!ht]
        \begin{center}
            \includegraphics[width=0.8\textwidth]{Lab4/charts/zad3/error.png}
        \end{center}
        %\caption{Schemat logiczny w edytorze LAD oraz tabela symboli}
        %\label{fig:picture}
    \end{figure}
    \FloatBarrier
\end{samepage}   
    
    

\subsection{Warunki brzegowe typu Neumanna}
\textbf{Postać ogólna zagadnień brzegowych rzędu II z warunkami brzegowymi typu Neumanna:}

\[
\begin{cases}
\vspace{0.1cm} 
\hspace{0,1cm} a(x) u'' + b(x)u' + c(x)u =f(x) \\
\vspace{0.1cm}
\hspace{0,1cm}u'|_{x=a} = \widetilde{u}_{a} \\
\hspace{0,1cm}u|_{x=b} = u_{b}
\end{cases}
\]
, gdzie:
$x\in[a,b]$
\\

\subsubsection{Cel ćwiczenia}
Kolejnym zadaniem było stworzenie algorytmów rozwiązujących zagadnienia brzegowe II rzędu z warunkami brzegowymi typu Neumanna.
\\\\
a) Pierwszego podejście polegało na wykorzystaniu schematu jednostronnego w celu aproksymacji pochodnej rzędu I występującej w warunku brzegowym typu Neumanna
\newpage
b) Drugie podejście polegało na dołączeniu dodatkowego węzła o indeksie (-1), który w rzeczywistości nie występuje w układzie równań
\\\\
c) Trzecie podejście polegało na wykorzystaniu jednostronnych schematów w węźle brzegowym
%\newpage

Do rozwiązania zostało podane następujące zagadnienie:
\[
\begin{cases}
\vspace{0.1cm} 
\hspace{0,1cm}u''-xu' + u=e^x(-x^2+x+2) \\
\vspace{0.1cm}
\hspace{0,1cm}u|_{x=0}=0 \\
\hspace{0,1cm}u'|_{x=1}=2e
\end{cases}
\]
, gdzie:

$x\in [0,1]$
\\
Rozwiązanie analityczne: $\widetilde{u}(x) = (x+1)e^x$

\subsubsection{Rozwiązanie}


\begin{samepage}
a)
\input{Lab4/zad4_1}
\end{samepage}
\newpage
\begin{samepage}
b)
\input{Lab4/zad4_2}
\end{samepage}
\newpage
\begin{samepage}
c)
\input{Lab4/zad4_3}
\end{samepage}