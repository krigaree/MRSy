\begin{Shaded}
\begin{Highlighting}[]
\CommentTok{%metoda Jacobiego}
\FunctionTok{clc}\NormalTok{, }\FunctionTok{clear} \FunctionTok{all}\NormalTok{; }\FunctionTok{tic}
\CommentTok{%funkcja}
\NormalTok{F = @(x,y) }\FloatTok{0}\NormalTok{;}
\CommentTok{%rozwiązanie analityczne}
\NormalTok{G = @(x,y) }\FunctionTok{log}\NormalTok{(x.^}\FloatTok{2}\NormalTok{+y.^}\FloatTok{2}\NormalTok{);}
\CommentTok{%przedział omega}
\NormalTok{xa=}\FloatTok{1}\NormalTok{; xb=}\FloatTok{2}\NormalTok{; yc=}\FloatTok{0}\NormalTok{; yd=}\FloatTok{1}\NormalTok{;}
\CommentTok{%warunki brzegowe}
\NormalTok{war1 = @(x) }\FloatTok{2}\NormalTok{*}\FunctionTok{log}\NormalTok{(x);}
\NormalTok{war2 = @(y) }\FunctionTok{log}\NormalTok{(y.^}\FloatTok{2}\NormalTok{+}\FloatTok{4}\NormalTok{);}
\NormalTok{war3 = @(x) }\FunctionTok{log}\NormalTok{(x.^}\FloatTok{2}\NormalTok{+}\FloatTok{1}\NormalTok{);}
\NormalTok{war4 = @(y) }\FunctionTok{log}\NormalTok{(y.^}\FloatTok{2}\NormalTok{+}\FloatTok{1}\NormalTok{);}
\CommentTok{%siatka}
\NormalTok{n=}\FloatTok{35}\NormalTok{; m=n;}
\NormalTok{lambda = }\FloatTok{0.25}\NormalTok{*(}\FunctionTok{cos}\NormalTok{(}\BaseNTok{pi}\NormalTok{/n)+}\FunctionTok{cos}\NormalTok{(}\BaseNTok{pi}\NormalTok{/m))^}\FloatTok{2}\NormalTok{;}
\NormalTok{omega = }\FloatTok{1}\NormalTok{+(lambda/(}\FloatTok{1}\NormalTok{+}\FunctionTok{sqrt}\NormalTok{(}\FloatTok{1}\NormalTok{-lambda))^}\FloatTok{2}\NormalTok{);}
\NormalTok{h=(xb-xa)/(n-}\FloatTok{1}\NormalTok{); x=}\FunctionTok{linspace}\NormalTok{(xa,xb,n+}\FloatTok{2}\NormalTok{); y=}\FunctionTok{linspace}\NormalTok{(yc,yd,m+}\FloatTok{2}\NormalTok{);}
\NormalTok{tol=}\FloatTok{1e-2}\NormalTok{; }\FunctionTok{error}\NormalTok{ = }\FloatTok{10}\NormalTok{; licznik=}\FloatTok{0}\NormalTok{; }
\CommentTok{%tworzenie macierzy}
\NormalTok{M0=}\FunctionTok{zeros}\NormalTok{(m,n)+}\FloatTok{1}\NormalTok{;}
\NormalTok{M1(}\FloatTok{1}\NormalTok{:n+}\FloatTok{2}\NormalTok{) = war1(x);}
\NormalTok{M2(}\FloatTok{1}\NormalTok{:m) = war2(y(}\FloatTok{2}\NormalTok{:}\FunctionTok{length}\NormalTok{(y)-}\FloatTok{1}\NormalTok{));}
\NormalTok{M3(}\FloatTok{1}\NormalTok{:n+}\FloatTok{2}\NormalTok{) = war3(x);}
\NormalTok{M4(}\FloatTok{1}\NormalTok{:m) = war4(y(}\FloatTok{2}\NormalTok{:}\FunctionTok{length}\NormalTok{(y)-}\FloatTok{1}\NormalTok{));}
\NormalTok{M0=[M4', M0, M2']; M0=[M1;M0;M3]; M=M0;}
\NormalTok{for }\BaseNTok{i}\NormalTok{=}\FloatTok{1}\NormalTok{:m+}\FloatTok{2}
\NormalTok{    for }\BaseNTok{j}\NormalTok{=}\FloatTok{1}\NormalTok{:n+}\FloatTok{2}
\NormalTok{        g(}\BaseNTok{i}\NormalTok{,}\BaseNTok{j}\NormalTok{) = G(x(}\BaseNTok{j}\NormalTok{),y(}\BaseNTok{i}\NormalTok{));}
\NormalTok{    end}
\NormalTok{end}
\NormalTok{while }\FunctionTok{error}\NormalTok{>tol}
\NormalTok{    for }\BaseNTok{i}\NormalTok{=}\FloatTok{2}\NormalTok{:m+}\FloatTok{1}
\NormalTok{        for }\BaseNTok{j}\NormalTok{=}\FloatTok{2}\NormalTok{:n+}\FloatTok{1}
\NormalTok{            M(}\BaseNTok{i}\NormalTok{,}\BaseNTok{j}\NormalTok{) =}\FloatTok{0.25}\NormalTok{*omega*(M(}\BaseNTok{i}\NormalTok{+}\FloatTok{1}\NormalTok{,}\BaseNTok{j}\NormalTok{)+M(}\BaseNTok{i}\NormalTok{-}\FloatTok{1}\NormalTok{,}\BaseNTok{j}\NormalTok{)\textbackslash{}+M(}\BaseNTok{i}\NormalTok{,}\BaseNTok{j}\NormalTok{+}\FloatTok{1}\NormalTok{)+M(}\BaseNTok{i}\NormalTok{,}\BaseNTok{j}\NormalTok{-}\FloatTok{1}\NormalTok{))\textbackslash{}}
\NormalTok{            +(}\FloatTok{1}\NormalTok{-omega)*M(}\BaseNTok{i}\NormalTok{,}\BaseNTok{j}\NormalTok{)-}\FloatTok{0.25}\NormalTok{*h^}\FloatTok{2}\NormalTok{*F(x(}\BaseNTok{j}\NormalTok{),y(}\BaseNTok{i}\NormalTok{));}
\NormalTok{        end}
\NormalTok{    end}
    \FunctionTok{error}\NormalTok{=}\FunctionTok{max}\NormalTok{(}\FunctionTok{max}\NormalTok{(}\FunctionTok{abs}\NormalTok{(M0-M)));}
\NormalTok{    M0=M; licznik = licznik+}\FloatTok{1}\NormalTok{;}
\NormalTok{end}
\CommentTok{%wykresy}
\NormalTok{[X,Y] = }\FunctionTok{meshgrid}\NormalTok{(x,y);}
\FunctionTok{subplot}\NormalTok{(}\FloatTok{1}\NormalTok{,}\FloatTok{2}\NormalTok{,}\FloatTok{1}\NormalTok{)}
\FunctionTok{surf}\NormalTok{(X,Y,M0)}
\FunctionTok{title}\NormalTok{(}\StringTok{'Metoda Numeryczna'}\NormalTok{)}
\FunctionTok{subplot}\NormalTok{(}\FloatTok{1}\NormalTok{,}\FloatTok{2}\NormalTok{,}\FloatTok{2}\NormalTok{)}
\FunctionTok{surf}\NormalTok{(X,Y,(G(X,Y)))}
\FunctionTok{title}\NormalTok{(}\StringTok{'Metoda Analityczna'}\NormalTok{)}
\NormalTok{licznik; }\FunctionTok{toc}
\end{Highlighting}
\end{Shaded}