\begin{Shaded}
\begin{Highlighting}[]
\CommentTok{%metoda Peacemanna-Rachforda}
\FunctionTok{tic}\NormalTok{; }\FunctionTok{clc}\NormalTok{; }\FunctionTok{clear} \FunctionTok{all}
\CommentTok{%funkcja}
\NormalTok{F = @(x,y) }\FloatTok{0}\NormalTok{;}
\CommentTok{%rozwiązanie analityczne}
\NormalTok{G = @(x,y) }\FunctionTok{log}\NormalTok{(x.^}\FloatTok{2}\NormalTok{+y.^}\FloatTok{2}\NormalTok{);}
\CommentTok{%przedział omega}
\NormalTok{xa=}\FloatTok{1}\NormalTok{; xb=}\FloatTok{2}\NormalTok{; yc=}\FloatTok{0}\NormalTok{; yd=}\FloatTok{1}\NormalTok{;}
\CommentTok{%warunki brzegowe}
\NormalTok{u1 = @(x) }\FloatTok{2}\NormalTok{*}\FunctionTok{log}\NormalTok{(x);}
\NormalTok{u2 = @(y) }\FunctionTok{log}\NormalTok{(y.^}\FloatTok{2}\NormalTok{+}\FloatTok{4}\NormalTok{);}
\NormalTok{u3 = @(x) }\FunctionTok{log}\NormalTok{(x.^}\FloatTok{2}\NormalTok{+}\FloatTok{1}\NormalTok{);}
\NormalTok{u4 = @(y) }\FunctionTok{log}\NormalTok{(y.^}\FloatTok{2}\NormalTok{+}\FloatTok{1}\NormalTok{);}
\CommentTok{%siatka}
\NormalTok{n=}\FloatTok{50}\NormalTok{; m=n;}
\NormalTok{h=(xb-xa)/(n+}\FloatTok{1}\NormalTok{); k=(yd-yc)/(m+}\FloatTok{1}\NormalTok{);}
\NormalTok{x=[xa:h:xb]; y=[yc:k:yd];}
\NormalTok{tol=}\FloatTok{1e-4}\NormalTok{; }\FunctionTok{error}\NormalTok{ = }\FloatTok{1}\NormalTok{; licznik=}\FloatTok{0}\NormalTok{; }
\CommentTok{%tworzenie macierzy}
\NormalTok{U = }\FunctionTok{zeros}\NormalTok{(m+}\FloatTok{2}\NormalTok{,n+}\FloatTok{2}\NormalTok{);}
\NormalTok{U(}\FloatTok{1}\NormalTok{,:) = u1(x);}
\NormalTok{U(m+}\FloatTok{2}\NormalTok{,:) = u3(x);}
\NormalTok{U(:,}\FloatTok{1}\NormalTok{) = u4(x);}
\NormalTok{U(:,n+}\FloatTok{2}\NormalTok{) = u2(x);}
\NormalTok{Uk=U;}
\NormalTok{if xb-xa > yd-yc mR = n; else mR = m; end}
\NormalTok{R = createR(mR+}\FloatTok{1}\NormalTok{);}
\NormalTok{lR = }\FunctionTok{length}\NormalTok{(R); }
\NormalTok{while }\FunctionTok{error}\NormalTok{>tol}
\NormalTok{      r = R(}\FunctionTok{mod}\NormalTok{(licznik,lR)+}\FloatTok{1}\NormalTok{);}
      \CommentTok{%step n -> n+1/2}
      \FunctionTok{step}\NormalTok{ = }\BaseNTok{true}\NormalTok{; iter = }\FunctionTok{length}\NormalTok{(y);}
\NormalTok{      for }\BaseNTok{i}\NormalTok{=}\FloatTok{2}\NormalTok{:iter-}\FloatTok{1}
\NormalTok{        Uk(}\BaseNTok{i}\NormalTok{, }\FloatTok{2}\NormalTok{:}\FunctionTok{length}\NormalTok{(x)-}\FloatTok{1}\NormalTok{) = doStep(}\BaseNTok{i}\NormalTok{, r, x, y, U, }\FunctionTok{step}\NormalTok{);}
\NormalTok{      end}
\NormalTok{      U = Uk;}
      \CommentTok{%step n+1/2 -> n+1}
      \FunctionTok{step}\NormalTok{ = }\BaseNTok{false}\NormalTok{; iter = }\FunctionTok{length}\NormalTok{(x);}
\NormalTok{      for }\BaseNTok{i}\NormalTok{=}\FloatTok{2}\NormalTok{:iter-}\FloatTok{1}
\NormalTok{        Uk(}\FloatTok{2}\NormalTok{:}\FunctionTok{length}\NormalTok{(y)-}\FloatTok{1}\NormalTok{, }\BaseNTok{i}\NormalTok{) = doStep(}\BaseNTok{i}\NormalTok{, r, x, y, U, }\FunctionTok{step}\NormalTok{);}
\NormalTok{      end}
\NormalTok{      licznik = licznik+}\FloatTok{1}\NormalTok{;}
      \FunctionTok{error}\NormalTok{ = }\FunctionTok{max}\NormalTok{(}\FunctionTok{max}\NormalTok{(}\FunctionTok{abs}\NormalTok{(Uk-U)));}
\NormalTok{      U=Uk;}
\NormalTok{end}
\CommentTok{%wykresy}
\NormalTok{[X,Y] = }\FunctionTok{meshgrid}\NormalTok{(x,y);}
\FunctionTok{subplot}\NormalTok{(}\FloatTok{1}\NormalTok{,}\FloatTok{2}\NormalTok{,}\FloatTok{1}\NormalTok{)}
\FunctionTok{surf}\NormalTok{(X,Y,U)}
\FunctionTok{title}\NormalTok{(}\StringTok{'Metoda Numeryczna'}\NormalTok{)}
\FunctionTok{subplot}\NormalTok{(}\FloatTok{1}\NormalTok{,}\FloatTok{2}\NormalTok{,}\FloatTok{2}\NormalTok{)}
\FunctionTok{surf}\NormalTok{(X,Y,(G(X,Y)))}
\FunctionTok{title}\NormalTok{(}\StringTok{'Metoda Analityczna'}\NormalTok{)}
\NormalTok{licznik; }\FunctionTok{toc}
\end{Highlighting}
\end{Shaded}