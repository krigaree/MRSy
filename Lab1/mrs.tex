\documentclass[a4paper]{article}

\usepackage{lmodern}
\usepackage{amssymb,amsmath}
\usepackage{ifxetex,ifluatex}
\usepackage{fixltx2e} % provides \textsubscript
\ifnum 0\ifxetex 1\fi\ifluatex 1\fi=0 % if pdftex
\usepackage[T1]{fontenc}
\usepackage[utf8]{inputenc}
\else % if luatex or xelatex
\ifxetex
\usepackage{mathspec}
\else
\usepackage{fontspec}
\fi
\defaultfontfeatures{Ligatures=TeX,Scale=MatchLowercase}
\fi
% use upquote if available, for straight quotes in verbatim environments
\IfFileExists{upquote.sty}{\usepackage{upquote}}{}
% use microtype if available
\IfFileExists{microtype.sty}{%
	\usepackage{microtype}
	\UseMicrotypeSet[protrusion]{basicmath} % disable protrusion for tt fonts
}{}
\usepackage[margin=1in]{geometry}
\usepackage{hyperref}
\hypersetup{unicode=true,
	pdftitle={lab1},
	pdfborder={0 0 0},
	breaklinks=true}
\urlstyle{same}  % don't use monospace font for urls
\usepackage{color}
\usepackage{fancyvrb}
\newcommand{\VerbBar}{|}
\newcommand{\VERB}{\Verb[commandchars=\\\{\}]}
\DefineVerbatimEnvironment{Highlighting}{Verbatim}{commandchars=\\\{\}}
% Add ',fontsize=\small' for more characters per line
\usepackage{framed}
\definecolor{shadecolor}{RGB}{248,248,248}
\newenvironment{Shaded}{\begin{snugshade}}{\end{snugshade}}
\newcommand{\KeywordTok}[1]{\textcolor[rgb]{0.13,0.29,0.53}{\textbf{{#1}}}}
\newcommand{\DataTypeTok}[1]{\textcolor[rgb]{0.13,0.29,0.53}{{#1}}}
\newcommand{\DecValTok}[1]{\textcolor[rgb]{0.00,0.00,0.81}{{#1}}}
\newcommand{\BaseNTok}[1]{\textcolor[rgb]{0.00,0.00,0.81}{{#1}}}
\newcommand{\FloatTok}[1]{\textcolor[rgb]{0.00,0.00,0.81}{{#1}}}
\newcommand{\ConstantTok}[1]{\textcolor[rgb]{0.00,0.00,0.00}{{#1}}}
\newcommand{\CharTok}[1]{\textcolor[rgb]{0.31,0.60,0.02}{{#1}}}
\newcommand{\SpecialCharTok}[1]{\textcolor[rgb]{0.00,0.00,0.00}{{#1}}}
\newcommand{\StringTok}[1]{\textcolor[rgb]{0.31,0.60,0.02}{{#1}}}
\newcommand{\VerbatimStringTok}[1]{\textcolor[rgb]{0.31,0.60,0.02}{{#1}}}
\newcommand{\SpecialStringTok}[1]{\textcolor[rgb]{0.31,0.60,0.02}{{#1}}}
\newcommand{\ImportTok}[1]{{#1}}
\newcommand{\CommentTok}[1]{\textcolor[rgb]{0.56,0.35,0.01}{\textit{{#1}}}}
\newcommand{\DocumentationTok}[1]{\textcolor[rgb]{0.56,0.35,0.01}{\textbf{\textit{{#1}}}}}
\newcommand{\AnnotationTok}[1]{\textcolor[rgb]{0.56,0.35,0.01}{\textbf{\textit{{#1}}}}}
\newcommand{\CommentVarTok}[1]{\textcolor[rgb]{0.56,0.35,0.01}{\textbf{\textit{{#1}}}}}
\newcommand{\OtherTok}[1]{\textcolor[rgb]{0.56,0.35,0.01}{{#1}}}
\newcommand{\FunctionTok}[1]{\textcolor[rgb]{0.00,0.00,0.00}{{#1}}}
\newcommand{\VariableTok}[1]{\textcolor[rgb]{0.00,0.00,0.00}{{#1}}}
\newcommand{\ControlFlowTok}[1]{\textcolor[rgb]{0.13,0.29,0.53}{\textbf{{#1}}}}
\newcommand{\OperatorTok}[1]{\textcolor[rgb]{0.81,0.36,0.00}{\textbf{{#1}}}}
\newcommand{\BuiltInTok}[1]{{#1}}
\newcommand{\ExtensionTok}[1]{{#1}}
\newcommand{\PreprocessorTok}[1]{\textcolor[rgb]{0.56,0.35,0.01}{\textit{{#1}}}}
\newcommand{\AttributeTok}[1]{\textcolor[rgb]{0.77,0.63,0.00}{{#1}}}
\newcommand{\RegionMarkerTok}[1]{{#1}}
\newcommand{\InformationTok}[1]{\textcolor[rgb]{0.56,0.35,0.01}{\textbf{\textit{{#1}}}}}
\newcommand{\WarningTok}[1]{\textcolor[rgb]{0.56,0.35,0.01}{\textbf{\textit{{#1}}}}}
\newcommand{\AlertTok}[1]{\textcolor[rgb]{0.94,0.16,0.16}{{#1}}}
\newcommand{\ErrorTok}[1]{\textcolor[rgb]{0.64,0.00,0.00}{\textbf{{#1}}}}
\newcommand{\NormalTok}[1]{{#1}}
\usepackage{graphicx,grffile}
\makeatletter
\def\maxwidth{\ifdim\Gin@nat@width>\linewidth\linewidth\else\Gin@nat@width\fi}
\def\maxheight{\ifdim\Gin@nat@height>\textheight\textheight\else\Gin@nat@height\fi}
\makeatother
% Scale images if necessary, so that they will not overflow the page
% margins by default, and it is still possible to overwrite the defaults
% using explicit options in \includegraphics[width, height, ...]{}
\setkeys{Gin}{width=\maxwidth,height=\maxheight,keepaspectratio}
\IfFileExists{parskip.sty}{%
	\usepackage{parskip}
}{% else
\setlength{\parindent}{0pt}
\setlength{\parskip}{6pt plus 2pt minus 1pt}
}
\setlength{\emergencystretch}{3em}  % prevent overfull lines
\providecommand{\tightlist}{%
	\setlength{\itemsep}{0pt}\setlength{\parskip}{0pt}}
\setcounter{secnumdepth}{2} %numbering!!!!
\setcounter{tocdepth}{2}
% Redefines (sub)paragraphs to behave more like sections
\ifx\paragraph\undefined\else
\let\oldparagraph\paragraph
\renewcommand{\paragraph}[1]{\oldparagraph{#1}\mbox{}}
\fi
\ifx\subparagraph\undefined\else
\let\oldsubparagraph\subparagraph
\renewcommand{\subparagraph}[1]{\oldsubparagraph{#1}\mbox{}}
\fi


\usepackage{polski}
\usepackage[english, polish]{babel}
\usepackage[utf8]{inputenc}
\usepackage{amsmath}
\usepackage{graphicx}
\usepackage[colorinlistoftodos]{todonotes}

\usepackage{fancyhdr}% http://ctan.org/pkg/fancyhdr

\fancyhf{}% Clear all headers/footers
\renewcommand{\headrulewidth}{0pt}% No header rule
\renewcommand{\footrulewidth}{0pt}% No footer rule
\fancyfoot[C]{\thepage}%

\fancypagestyle{firstpage}{
    \fancyfoot[C]{
    Prowadzący: mgr inż. Marcin Stasiak
    }
    }

\title{\normalsize \textsc{
        \Large{POLITECHNIKA POZNAŃSKA}\\
		\large{Wydział Elektryczny}\\
		\normalsize
		Instytut Matematyki\\
		Zakład Zastosowań Matematyki
		}
    \\[4.0cm]
    \Large{Praca semestralna}\\
    [0.5cm]
    \Huge{Metoda Różnic Skończonych}
    \normalsize \vspace*{5\baselineskip}
    }
\author{Mateusz Talarski\\i\\Piotr Wyrwiński}

\date{}
\begin{document}
	\maketitle
    \thispagestyle{firstpage}
	\newpage
	
	\tableofcontents
	\newpage
	
	
	\section{Macierze Kroneckera}
	\subsection{Iloczyn Kroneckera}
		%\item \textbf{Iloczyn Kroneckera}
		
		Jeżeli \textbf{A} jest macierzą o wymiarze $m \times n$, oraz \textbf{B} jest $p \times q$, to \textbf{iloczynem Kroneckera} $\mathbf{A \otimes B}$ nazywamy macierz blokową $mp \times nq$
		
	$$\mathbf{A \otimes B} = \begin{bmatrix} a_{11} \mathbf{B} & \cdots & a_{1n}\mathbf{B} \\ \vdots & \ddots & \vdots \\ a_{m1} \mathbf{B} & \cdots & a_{mn} \mathbf{B} \end{bmatrix}$$
	
		\textbf{Przykład}
	$$\begin{bmatrix}1&2\\3&4\\\end{bmatrix}\otimes {\begin{bmatrix}0&5\\6&7\\\end{bmatrix}}={\begin{bmatrix}1\cdot {\begin{bmatrix}0&5\\6&7\\\end{bmatrix}}&2\cdot {\begin{bmatrix}0&5\\6&7\\\end{bmatrix}}\\3\cdot {\begin{bmatrix}0&5\\6&7\\\end{bmatrix}}&4\cdot {\begin{bmatrix}0&5\\6&7\\\end{bmatrix}}\\\end{bmatrix}}={\begin{bmatrix}1\cdot 0&1\cdot 5&2\cdot 0&2\cdot 5\\1\cdot 6&1\cdot 7&2\cdot 6&2\cdot 7\\3\cdot 0&3\cdot 5&4\cdot 0&4\cdot 5\\3\cdot 6&3\cdot 7&4\cdot 6&4\cdot 7\\\end{bmatrix}}={\begin{bmatrix}0&5&0&10\\6&7&12&14\\0&15&0&20\\18&21&24&28\end{bmatrix}}$$
	
\subsection{Cel zajęć}
Na laboratorium przy pomocy iloczyny Kroneckera oraz zadanych macierzy A, B, C, D mieliśmy utworzyć odpowienie macierze. Wyniki przedstawiamy poniżej:




\begin{Shaded}
\begin{Highlighting}[]
\NormalTok{n=}\FloatTok{6}\NormalTok{;}
\NormalTok{A=[}\FloatTok{2}\NormalTok{, -}\FloatTok{1}\NormalTok{;-}\FloatTok{1}\NormalTok{,}\FloatTok{2}\NormalTok{];}
\NormalTok{B=[}\FloatTok{0}\NormalTok{,-}\FloatTok{1}\NormalTok{;}\FloatTok{2}\NormalTok{,}\FloatTok{3}\NormalTok{];}
\NormalTok{C=[}\FloatTok{1}\NormalTok{,}\FloatTok{1}\NormalTok{;}\FloatTok{1}\NormalTok{,}\FloatTok{2}\NormalTok{];}
\end{Highlighting}
\end{Shaded}
        

\begin{Shaded}
\begin{Highlighting}[]       
\CommentTok{%E1}
\NormalTok{E1 = }\FunctionTok{kron}\NormalTok{(}\FunctionTok{eye}\NormalTok{(n),A);}
\end{Highlighting}
\end{Shaded}           

\begin{Shaded}
\begin{Highlighting}[]
\CommentTok{%E2}
\NormalTok{v = [}\FloatTok{1}\NormalTok{;}\FloatTok{2}\NormalTok{*}\FunctionTok{diag}\NormalTok{(}\FunctionTok{eye}\NormalTok{(n-}\FloatTok{2}\NormalTok{));}\FloatTok{1}\NormalTok{];}
\NormalTok{E2 = }\FunctionTok{kron}\NormalTok{(}\FunctionTok{diag}\NormalTok{(v),A);}
\end{Highlighting}
\end{Shaded}
                
\begin{Shaded}
\begin{Highlighting}[]
\CommentTok{%E3}
\NormalTok{v = [}\FloatTok{1}\NormalTok{;-}\FloatTok{1}\NormalTok{*}\FunctionTok{diag}\NormalTok{(}\FunctionTok{eye}\NormalTok{(n-}\FloatTok{2}\NormalTok{));}\FloatTok{1}\NormalTok{];}
\NormalTok{E3 = }\FunctionTok{kron}\NormalTok{(}\FunctionTok{diag}\NormalTok{(v),A) +\textbackslash{}}
\FunctionTok{kron}\NormalTok{(}\FunctionTok{diag}\NormalTok{(}\FunctionTok{diag}\NormalTok{(}\FunctionTok{eye}\NormalTok{(n-}\FloatTok{1}\NormalTok{)),}\FloatTok{1}\NormalTok{),B) +\textbackslash{}}
\FunctionTok{kron}\NormalTok{(}\FunctionTok{diag}\NormalTok{(}\FunctionTok{diag}\NormalTok{(}\FunctionTok{eye}\NormalTok{(n-}\FloatTok{1}\NormalTok{)),-}\FloatTok{1}\NormalTok{),B);}
\end{Highlighting}
\end{Shaded}

\begin{Shaded}
\begin{Highlighting}[]
\CommentTok{%E4  }
\NormalTok{v = [}\FloatTok{1}\NormalTok{;}\FloatTok{2}\NormalTok{*}\FunctionTok{diag}\NormalTok{(}\FunctionTok{eye}\NormalTok{(n-}\FloatTok{2}\NormalTok{));}\FloatTok{1}\NormalTok{];}
\NormalTok{E4 = }\FunctionTok{kron}\NormalTok{(}\FunctionTok{diag}\NormalTok{(v),A) +\textbackslash{}}
\FunctionTok{kron}\NormalTok{(}\FunctionTok{diag}\NormalTok{(}\FunctionTok{diag}\NormalTok{(}\FunctionTok{eye}\NormalTok{(n-}\FloatTok{1}\NormalTok{)),}\FloatTok{1}\NormalTok{),B) +\textbackslash{}}
\FunctionTok{kron}\NormalTok{(}\FunctionTok{diag}\NormalTok{(}\FunctionTok{diag}\NormalTok{(}\FunctionTok{eye}\NormalTok{(n-}\FloatTok{1}\NormalTok{)),-}\FloatTok{1}\NormalTok{),B');}
\end{Highlighting}
\end{Shaded}
                        
\begin{Shaded}
\begin{Highlighting}[]
\CommentTok{%E5}
\NormalTok{E5=}\FunctionTok{kron}\NormalTok{(}\FunctionTok{eye}\NormalTok{(n),A) +\textbackslash{}}
\FunctionTok{kron}\NormalTok{(}\FunctionTok{diag}\NormalTok{(}\FunctionTok{diag}\NormalTok{(}\FunctionTok{eye}\NormalTok{(n-}\FloatTok{2}\NormalTok{)),}\FloatTok{2}\NormalTok{),B') +\textbackslash{}}
\FunctionTok{kron}\NormalTok{(}\FunctionTok{diag}\NormalTok{(}\FunctionTok{diag}\NormalTok{(}\FunctionTok{eye}\NormalTok{(n-}\FloatTok{2}\NormalTok{)),-}\FloatTok{2}\NormalTok{),B);}
\end{Highlighting}
\end{Shaded}
\begin{samepage}
\begin{Shaded}
\begin{Highlighting}[]
\CommentTok{%E6  }
\NormalTok{v = [}\FloatTok{1}\NormalTok{;}\FunctionTok{zeros}\NormalTok{(n-}\FloatTok{2}\NormalTok{,}\FloatTok{1}\NormalTok{);}\FloatTok{1}\NormalTok{];}
\NormalTok{v1 = [}\FloatTok{0}\NormalTok{;}\FunctionTok{diag}\NormalTok{(}\FunctionTok{eye}\NormalTok{(n-}\FloatTok{2}\NormalTok{));}\FloatTok{0}\NormalTok{];}
\NormalTok{E6 = }\FunctionTok{kron}\NormalTok{(}\FunctionTok{diag}\NormalTok{(v),C)+ }\FunctionTok{kron}\NormalTok{(}\FunctionTok{diag}\NormalTok{(v1),A) +\textbackslash{}}
\FunctionTok{kron}\NormalTok{(}\FunctionTok{diag}\NormalTok{(}\FunctionTok{diag}\NormalTok{(}\FunctionTok{eye}\NormalTok{(n-}\FloatTok{2}\NormalTok{)),}\FloatTok{2}\NormalTok{),}\FloatTok{3}\NormalTok{*B') +\textbackslash{}}
\FunctionTok{kron}\NormalTok{(}\FunctionTok{diag}\NormalTok{(}\FunctionTok{diag}\NormalTok{(}\FunctionTok{eye}\NormalTok{(n-}\FloatTok{2}\NormalTok{)),-}\FloatTok{2}\NormalTok{),}\FloatTok{2}\NormalTok{*B) +\textbackslash{}}
\NormalTok{flip(}\FunctionTok{kron}\NormalTok{(}\FunctionTok{diag}\NormalTok{(v), flip(}\FunctionTok{eye}\NormalTok{(}\FloatTok{2}\NormalTok{))));}
dsa\\dsa\\dsa\\dsadsa\\dsada\\
\end{Highlighting}
\end{Shaded}
\end{samepage}

\begin{Shaded}
\begin{Highlighting}[]
\CommentTok{%E7}
\NormalTok{v=[}\FloatTok{1}\NormalTok{;}\FunctionTok{zeros}\NormalTok{(n-}\FloatTok{2}\NormalTok{,}\FloatTok{1}\NormalTok{);}\FloatTok{1}\NormalTok{];}
\NormalTok{v1=[}\FloatTok{0}\NormalTok{;}\FloatTok{1}\NormalTok{;}\FunctionTok{zeros}\NormalTok{(n-}\FloatTok{4}\NormalTok{,}\FloatTok{1}\NormalTok{);}\FloatTok{1}\NormalTok{;}\FloatTok{0}\NormalTok{];}
\NormalTok{v2=[}\FloatTok{0}\NormalTok{;}\FloatTok{0}\NormalTok{;}\FloatTok{2}\NormalTok{*}\FunctionTok{diag}\NormalTok{(}\FunctionTok{eye}\NormalTok{(n-}\FloatTok{4}\NormalTok{));}\FloatTok{0}\NormalTok{;}\FloatTok{0}\NormalTok{];}
\NormalTok{E7=}\FunctionTok{kron}\NormalTok{(}\FunctionTok{diag}\NormalTok{(v),A) +\textbackslash{}}
\FunctionTok{kron}\NormalTok{(}\FunctionTok{diag}\NormalTok{(v1),B) +\textbackslash{}}
\FunctionTok{kron}\NormalTok{(}\FunctionTok{diag}\NormalTok{(v2),B) +\textbackslash{}}
\FunctionTok{kron}\NormalTok{(}\FunctionTok{diag}\NormalTok{(}\FunctionTok{diag}\NormalTok{(}\FunctionTok{eye}\NormalTok{(n-}\FloatTok{2}\NormalTok{)),}\FloatTok{2}\NormalTok{),C) +\textbackslash{}}
\FunctionTok{kron}\NormalTok{(}\FunctionTok{diag}\NormalTok{(}\FunctionTok{diag}\NormalTok{(}\FunctionTok{eye}\NormalTok{(n-}\FloatTok{2}\NormalTok{)),-}\FloatTok{2}\NormalTok{),C');}
\end{Highlighting}
\end{Shaded}

\begin{Shaded}
\begin{Highlighting}[]
\CommentTok{%E8}
\NormalTok{v=[}\FloatTok{1}\NormalTok{;}\FunctionTok{zeros}\NormalTok{(n-}\FloatTok{2}\NormalTok{,}\FloatTok{1}\NormalTok{);}\FloatTok{1}\NormalTok{];}
\NormalTok{v1=[}\FloatTok{0}\NormalTok{;}\FloatTok{2}\NormalTok{;}\FunctionTok{zeros}\NormalTok{(n-}\FloatTok{4}\NormalTok{,}\FloatTok{1}\NormalTok{);}\FloatTok{2}\NormalTok{;}\FloatTok{0}\NormalTok{];}
\NormalTok{v2=[}\FloatTok{0}\NormalTok{;}\FloatTok{0}\NormalTok{;}\FunctionTok{diag}\NormalTok{(}\FunctionTok{eye}\NormalTok{(n-}\FloatTok{4}\NormalTok{));}\FloatTok{0}\NormalTok{;}\FloatTok{0}\NormalTok{];}

\NormalTok{E8=}\FunctionTok{kron}\NormalTok{(}\FunctionTok{diag}\NormalTok{(v),A) +\textbackslash{}}
\FunctionTok{kron}\NormalTok{(}\FunctionTok{diag}\NormalTok{(v1),B) + }\FunctionTok{kron}\NormalTok{(}\FunctionTok{diag}\NormalTok{(v2),C) +\textbackslash{}}
\FunctionTok{kron}\NormalTok{(}\FunctionTok{diag}\NormalTok{(}\FunctionTok{diag}\NormalTok{(}\FunctionTok{eye}\NormalTok{(n-}\FloatTok{2}\NormalTok{)),}\FloatTok{2}\NormalTok{),}\FunctionTok{eye}\NormalTok{(}\FloatTok{2}\NormalTok{)) +\textbackslash{}}
\FunctionTok{kron}\NormalTok{(}\FunctionTok{diag}\NormalTok{(}\FunctionTok{diag}\NormalTok{(}\FunctionTok{eye}\NormalTok{(n-}\FloatTok{2}\NormalTok{)),-}\FloatTok{2}\NormalTok{),}\FunctionTok{eye}\NormalTok{(}\FloatTok{2}\NormalTok{))+\textbackslash{}}
\NormalTok{flip(}\FunctionTok{kron}\NormalTok{(}\FunctionTok{diag}\NormalTok{(v),flip(A)));}
\end{Highlighting}
\end{Shaded}

\begin{verbatim}
## E1 =
## 
##    2  -1   0   0   0   0   0   0   0   0   0   0
##   -1   2   0   0   0   0   0   0   0   0   0   0
##    0   0   2  -1   0   0   0   0   0   0   0   0
##    0   0  -1   2   0   0   0   0   0   0   0   0
##    0   0   0   0   2  -1   0   0   0   0   0   0
##    0   0   0   0  -1   2   0   0   0   0   0   0
##    0   0   0   0   0   0   2  -1   0   0   0   0
##    0   0   0   0   0   0  -1   2   0   0   0   0
##    0   0   0   0   0   0   0   0   2  -1   0   0
##    0   0   0   0   0   0   0   0  -1   2   0   0
##    0   0   0   0   0   0   0   0   0   0   2  -1
##    0   0   0   0   0   0   0   0   0   0  -1   2
## 
## E2 =
## 
##    2  -1   0   0   0   0   0   0   0   0   0   0
##   -1   2   0   0   0   0   0   0   0   0   0   0
##    0   0   4  -2   0   0   0   0   0   0   0   0
##    0   0  -2   4   0   0   0   0   0   0   0   0
##    0   0   0   0   4  -2   0   0   0   0   0   0
##    0   0   0   0  -2   4   0   0   0   0   0   0
##    0   0   0   0   0   0   4  -2   0   0   0   0
##    0   0   0   0   0   0  -2   4   0   0   0   0
##    0   0   0   0   0   0   0   0   4  -2   0   0
##    0   0   0   0   0   0   0   0  -2   4   0   0
##    0   0   0   0   0   0   0   0   0   0   2  -1
##    0   0   0   0   0   0   0   0   0   0  -1   2
## 
## E3 =
## 
##    2  -1   0  -1   0   0   0   0   0   0   0   0
##   -1   2   2   3   0   0   0   0   0   0   0   0
##    0  -1  -2   1   0  -1   0   0   0   0   0   0
##    2   3   1  -2   2   3   0   0   0   0   0   0
##    0   0   0  -1  -2   1   0  -1   0   0   0   0
##    0   0   2   3   1  -2   2   3   0   0   0   0
##    0   0   0   0   0  -1  -2   1   0  -1   0   0
##    0   0   0   0   2   3   1  -2   2   3   0   0
##    0   0   0   0   0   0   0  -1  -2   1   0  -1
##    0   0   0   0   0   0   2   3   1  -2   2   3
##    0   0   0   0   0   0   0   0   0  -1   2  -1
##    0   0   0   0   0   0   0   0   2   3  -1   2
## 
## E4 =
## 
##    2  -1   0  -1   0   0   0   0   0   0   0   0
##   -1   2   2   3   0   0   0   0   0   0   0   0
##    0   2   4  -2   0  -1   0   0   0   0   0   0
##   -1   3  -2   4   2   3   0   0   0   0   0   0
##    0   0   0   2   4  -2   0  -1   0   0   0   0
##    0   0  -1   3  -2   4   2   3   0   0   0   0
##    0   0   0   0   0   2   4  -2   0  -1   0   0
##    0   0   0   0  -1   3  -2   4   2   3   0   0
##    0   0   0   0   0   0   0   2   4  -2   0  -1
##    0   0   0   0   0   0  -1   3  -2   4   2   3
##    0   0   0   0   0   0   0   0   0   2   2  -1
##    0   0   0   0   0   0   0   0  -1   3  -1   2
## 
## E5 =
## 
##    2  -1   0   0   0   2   0   0   0   0   0   0
##   -1   2   0   0  -1   3   0   0   0   0   0   0
##    0   0   2  -1   0   0   0   2   0   0   0   0
##    0   0  -1   2   0   0  -1   3   0   0   0   0
##    0  -1   0   0   2  -1   0   0   0   2   0   0
##    2   3   0   0  -1   2   0   0  -1   3   0   0
##    0   0   0  -1   0   0   2  -1   0   0   0   2
##    0   0   2   3   0   0  -1   2   0   0  -1   3
##    0   0   0   0   0  -1   0   0   2  -1   0   0
##    0   0   0   0   2   3   0   0  -1   2   0   0
##    0   0   0   0   0   0   0  -1   0   0   2  -1
##    0   0   0   0   0   0   2   3   0   0  -1   2
## 
## E6 =
## 
##    1   1   0   0   0   6   0   0   0   0   1   0
##    1   2   0   0  -3   9   0   0   0   0   0   1
##    0   0   2  -1   0   0   0   6   0   0   0   0
##    0   0  -1   2   0   0  -3   9   0   0   0   0
##    0  -2   0   0   2  -1   0   0   0   6   0   0
##    4   6   0   0  -1   2   0   0  -3   9   0   0
##    0   0   0  -2   0   0   2  -1   0   0   0   6
##    0   0   4   6   0   0  -1   2   0   0  -3   9
##    0   0   0   0   0  -2   0   0   2  -1   0   0
##    0   0   0   0   4   6   0   0  -1   2   0   0
##    1   0   0   0   0   0   0  -2   0   0   1   1
##    0   1   0   0   0   0   4   6   0   0   1   2
## 
## E7 =
## 
##    2  -1   0   0   1   1   0   0   0   0   0   0
##   -1   2   0   0   1   2   0   0   0   0   0   0
##    0   0   0  -1   0   0   1   1   0   0   0   0
##    0   0   2   3   0   0   1   2   0   0   0   0
##    1   1   0   0   0  -2   0   0   1   1   0   0
##    1   2   0   0   4   6   0   0   1   2   0   0
##    0   0   1   1   0   0   0  -2   0   0   1   1
##    0   0   1   2   0   0   4   6   0   0   1   2
##    0   0   0   0   1   1   0   0   0  -1   0   0
##    0   0   0   0   1   2   0   0   2   3   0   0
##    0   0   0   0   0   0   1   1   0   0   2  -1
##    0   0   0   0   0   0   1   2   0   0  -1   2
## 
## E8 =
## 
##    2  -1   0   0   1   0   0   0   0   0   2  -1
##   -1   2   0   0   0   1   0   0   0   0  -1   2
##    0   0   0  -2   0   0   1   0   0   0   0   0
##    0   0   4   6   0   0   0   1   0   0   0   0
##    1   0   0   0   1   1   0   0   1   0   0   0
##    0   1   0   0   1   2   0   0   0   1   0   0
##    0   0   1   0   0   0   1   1   0   0   1   0
##    0   0   0   1   0   0   1   2   0   0   0   1
##    0   0   0   0   1   0   0   0   0  -2   0   0
##    0   0   0   0   0   1   0   0   4   6   0   0
##    2  -1   0   0   0   0   1   0   0   0   2  -1
##   -1   2   0   0   0   0   0   1   0   0  -1   2
\end{verbatim}


		

	
	
	\section{Ejercicios}
	
	\begin{enumerate}
		\item\textbf{1.Para deslizar un armario pesado por el piso a una rapidez constante, tu ejerces una fuerza horizontal de 600 N. ¿La fuerza de fricción entre el armario y el piso es mayor que, menor que o igual a 600 N? Defiende tu respuesta.}
		
		R/La fuerza de fricción tiene que ser igual a 600N para que esté en equilibrio y la rapidez del objeto sea constante.
		
		\item\textbf{Una jarra vacía con peso W descansa sobre una mesa. ¿Cuál es la fuerza de soporte que la mesa 
			ejerce sobre la jarra? ¿Cuál es la fuerza de soporte cuando se vierte en la jarra agua que pesa W?}
		
		R/Cuando la jarra esta vacía en la mesa se produce la fuerza normal para que este en equilibrio, cuando se vierte agua a la jarra la normal cambiara para igualar el peso y que el objeto quede en equilibrio.
		\newpage
		\item\textbf{Un coche recorre cierta carretera con una rapidez promedio de 40 km/h, y regresa por ella con 
			una rapidez promedio de 60 km/h. Calcula la rapidez promedio en el viaje redondo. (¡No es 50 
			km/h!)}
		
		$$V_{p}=\frac{40Km}{h} + \frac{60Km}{h} = \frac{100Km}{h}$$ 
		$$t_{1}=\frac{100Km}{40Km/h} = 2.5h\; t_{2}= \frac{100Km}{60Km/h} = 1.667h$$ 
		$$t_{p}=\frac{2.5h +1.667h}{2}= 2.0835h$$ 
		$$V_{m}=\frac{100Km}{2.0835h}= 47.9\frac{Km}{h} = 48\frac{Km}{h}$$ 
		La rapidex promedio es de $48\frac{Km}{h}$
		
		\item\textbf{Un automóvil tarda 10 s en pasar de v = 0 a v = 25 m/s con una aceleración aproximadamente 
			constante. Si deseas calcular la distancia recorrida con la ecuación \textbf{$d=\frac{1}{2}at^2$}, ¿qué valor usarías 
			en a?} 
		
		$$d=\frac{1}{2}at^2$$
		$$d=\frac{1}{2}(5m/s^2)(10s)^2$$
		$$d=(0.5)(5m/s^2)(100s^2)$$
		$$d=250m$$
		$$v=\frac{m}{s}$$
		$$v=\frac{250m}{10s}$$
		$$v=25\frac{m}{s}$$
		utilizaria 5 ya que con el 5 la distancia me daria 250m, y al utilizar esa distancia para sacar la rapidez el resultado nos daria $25\frac{m}{s}$
		
		\item\textbf{ Al acelerar cerca del final de una carrera, un corredor de 60 kg de masa pasa de una rapidez de 6 
			m/s a otra de 7 m/s en 2 s. 
			a. ¿Cuál es la aceleración promedio del corredor durante este tiempo? 
			b. Para aumentar su rapidez, el corredor produce una fuerza sobre el suelo dirigida hacia 
			atrás, y en consecuencia el suelo lo impulsa hacia adelante y proporciona la fuerza 
			necesaria para la aceleración. Calcula esta fuerza promedio.}
		
		R/ a) $$a_{p}=\frac{v_{f}-v_{i}}{t_{f}-t_{i}}$$
		$$a_{p}=\frac{7m/s-6m/s}{2s}$$
		$$a_{p}=\frac{1m/s}{2s}$$
		$$a_{p}=0.5m/s^2$$
		
		b) $$F=ma$$
		$$F=(60kg)(0.5m/s^2)$$
		$$F=30N$$
		
		\item\textbf{Se ve que dos cajas aceleran igual cuando se aplica una fuerza F a la primera, y se aplica 4F a la 
			segunda. ¿Cuál es la relación de sus masas?} 
		
		R/La masa de la primera caja es menor a la de la segunda caja que es 4 veces mayor por lo tanto se aplica mas fuerza en esta, pero siguen teniendo la misma aceleracíon.
		
		\item\textbf{Vas remando en una canoa, a 4 km/h tratando de cruzar directamente un río que corre a 3 km/h, 
			como se ve en la figura. a) ¿Cuál es la rapidez resultante de la canoa relativa a la orilla? b) ¿En 
			aproximadamente qué dirección debería remarse la canoa para que llegue a la otra orilla y su 
			trayectoria sea perpendicular al río?} 
		
		\begin{figure}[!ht]  
			
			
		\end{figure}
		
		R/ $$R=\sqrt{x^2+y^2}$$
		$$R=\sqrt{(4)^2+(3)^2}$$
		$$R=\sqrt{16+9}$$
		$$R=\sqrt{25}$$
		$$R=5\frac{Km}{h}$$
		Va a $5\frac{Km}{h}$ al sureste.
		
		\newpage
		
		\item\textbf{Un avión cuya rapidez normal es 100 km/h, pasa por un viento cruzado del oeste hacia el este de 
			100 km/h. Calcula su velocidad con respecto al suelo, cuando su proa apunta al norte, dentro del 
			viento cruzado.}
		
		R/ $$R=\sqrt{x^2+y^2}$$
		$$R=\sqrt{(100)^2+(100)^2}$$
		$$R=\sqrt{10000+10000}$$
		$$R=\sqrt{20000}$$
		$$R=141.42\frac{Km}{h}$$
		El avion va a una velocidad de $141.42\frac{Km}{h}$ hacia el noreste.
		
		\item\textbf{Superman llega a un asteroide en el espacio exterior y lo lanza a 800 m/s, tan rápido como una 
			bala. El asteroide es 1,000 veces más masivo que Superman. En los dibujos animados, se ve que 
			Superman queda inmóvil después del lanzamiento. Si entra la física en este caso, ¿cuál sería su 
			velocidad de retroceso?}
		
		R/ 
		
		
		\item\textbf{. Una locomotora diesel pesa cuatro veces más que un furgón de carga. Si la locomotora rueda a 5 
			km/h y choca contra un furgón que inicialmente está en reposo, ¿con qué rapidez siguen rodando 
			los dos después de acoplarse?}
		
		R/$$(mv_{neto})_{antes}=(mv_{neto})_{despues}$$
		$$(4m.5Km/h)+(m.0)=(4m+m)v$$
		$$\frac{20kg.Km/h}{5m}=v$$
		$$v=4\frac{Km}{h}$$
		
		\item\textbf{La energía que obtenemos del metabolismo puede efectuar trabajo y generar calor. 
			a. ¿Cuál es la eficiencia mecánica de una persona relativamente inactiva que gasta 100 W 
			de potencia para producir aproximadamente 1 W de potencia en forma de trabajo, 
			mientras genera más o menos 99 W de calor? 
			b. ¿Cuál es la eficiencia mecánica de un ciclista que, en una ráfaga de esfuerzo produce 100 
			W de potencia mecánica con 1,000 W de potencia metabólica?}
		
		R/ 
		
		
		
		R/ La energía cinética es la mitad después del choque por su rapides que es menor, y la energía se convierte en calor por la fricción producida.
		
		
		
	\end{enumerate}
	
	
\end{document}