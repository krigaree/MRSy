\documentclass[]{article}
\usepackage{lmodern}
\usepackage{amssymb,amsmath}
\usepackage{ifxetex,ifluatex}
\usepackage{fixltx2e} % provides \textsubscript
\ifnum 0\ifxetex 1\fi\ifluatex 1\fi=0 % if pdftex
  \usepackage[T1]{fontenc}
  \usepackage[utf8]{inputenc}
\else % if luatex or xelatex
  \ifxetex
    \usepackage{mathspec}
  \else
    \usepackage{fontspec}
  \fi
  \defaultfontfeatures{Ligatures=TeX,Scale=MatchLowercase}
\fi
% use upquote if available, for straight quotes in verbatim environments
\IfFileExists{upquote.sty}{\usepackage{upquote}}{}
% use microtype if available
\IfFileExists{microtype.sty}{%
\usepackage{microtype}
\UseMicrotypeSet[protrusion]{basicmath} % disable protrusion for tt fonts
}{}
\usepackage[margin=1in]{geometry}
\usepackage{hyperref}
\hypersetup{unicode=true,
            pdftitle={R\_M\_B},
            pdfborder={0 0 0},
            breaklinks=true}
\urlstyle{same}  % don't use monospace font for urls
\usepackage{color}
\usepackage{fancyvrb}
\newcommand{\VerbBar}{|}
\newcommand{\VERB}{\Verb[commandchars=\\\{\}]}
\DefineVerbatimEnvironment{Highlighting}{Verbatim}{commandchars=\\\{\}}
% Add ',fontsize=\small' for more characters per line
\usepackage{framed}
\definecolor{shadecolor}{RGB}{248,248,248}
\newenvironment{Shaded}{\begin{snugshade}}{\end{snugshade}}
\newcommand{\KeywordTok}[1]{\textcolor[rgb]{0.13,0.29,0.53}{\textbf{#1}}}
\newcommand{\DataTypeTok}[1]{\textcolor[rgb]{0.13,0.29,0.53}{#1}}
\newcommand{\DecValTok}[1]{\textcolor[rgb]{0.00,0.00,0.81}{#1}}
\newcommand{\BaseNTok}[1]{\textcolor[rgb]{0.00,0.00,0.81}{#1}}
\newcommand{\FloatTok}[1]{\textcolor[rgb]{0.00,0.00,0.81}{#1}}
\newcommand{\ConstantTok}[1]{\textcolor[rgb]{0.00,0.00,0.00}{#1}}
\newcommand{\CharTok}[1]{\textcolor[rgb]{0.31,0.60,0.02}{#1}}
\newcommand{\SpecialCharTok}[1]{\textcolor[rgb]{0.00,0.00,0.00}{#1}}
\newcommand{\StringTok}[1]{\textcolor[rgb]{0.31,0.60,0.02}{#1}}
\newcommand{\VerbatimStringTok}[1]{\textcolor[rgb]{0.31,0.60,0.02}{#1}}
\newcommand{\SpecialStringTok}[1]{\textcolor[rgb]{0.31,0.60,0.02}{#1}}
\newcommand{\ImportTok}[1]{#1}
\newcommand{\CommentTok}[1]{\textcolor[rgb]{0.56,0.35,0.01}{\textit{#1}}}
\newcommand{\DocumentationTok}[1]{\textcolor[rgb]{0.56,0.35,0.01}{\textbf{\textit{#1}}}}
\newcommand{\AnnotationTok}[1]{\textcolor[rgb]{0.56,0.35,0.01}{\textbf{\textit{#1}}}}
\newcommand{\CommentVarTok}[1]{\textcolor[rgb]{0.56,0.35,0.01}{\textbf{\textit{#1}}}}
\newcommand{\OtherTok}[1]{\textcolor[rgb]{0.56,0.35,0.01}{#1}}
\newcommand{\FunctionTok}[1]{\textcolor[rgb]{0.00,0.00,0.00}{#1}}
\newcommand{\VariableTok}[1]{\textcolor[rgb]{0.00,0.00,0.00}{#1}}
\newcommand{\ControlFlowTok}[1]{\textcolor[rgb]{0.13,0.29,0.53}{\textbf{#1}}}
\newcommand{\OperatorTok}[1]{\textcolor[rgb]{0.81,0.36,0.00}{\textbf{#1}}}
\newcommand{\BuiltInTok}[1]{#1}
\newcommand{\ExtensionTok}[1]{#1}
\newcommand{\PreprocessorTok}[1]{\textcolor[rgb]{0.56,0.35,0.01}{\textit{#1}}}
\newcommand{\AttributeTok}[1]{\textcolor[rgb]{0.77,0.63,0.00}{#1}}
\newcommand{\RegionMarkerTok}[1]{#1}
\newcommand{\InformationTok}[1]{\textcolor[rgb]{0.56,0.35,0.01}{\textbf{\textit{#1}}}}
\newcommand{\WarningTok}[1]{\textcolor[rgb]{0.56,0.35,0.01}{\textbf{\textit{#1}}}}
\newcommand{\AlertTok}[1]{\textcolor[rgb]{0.94,0.16,0.16}{#1}}
\newcommand{\ErrorTok}[1]{\textcolor[rgb]{0.64,0.00,0.00}{\textbf{#1}}}
\newcommand{\NormalTok}[1]{#1}
\usepackage{graphicx,grffile}
\makeatletter
\def\maxwidth{\ifdim\Gin@nat@width>\linewidth\linewidth\else\Gin@nat@width\fi}
\def\maxheight{\ifdim\Gin@nat@height>\textheight\textheight\else\Gin@nat@height\fi}
\makeatother
% Scale images if necessary, so that they will not overflow the page
% margins by default, and it is still possible to overwrite the defaults
% using explicit options in \includegraphics[width, height, ...]{}
\setkeys{Gin}{width=\maxwidth,height=\maxheight,keepaspectratio}
\IfFileExists{parskip.sty}{%
\usepackage{parskip}
}{% else
\setlength{\parindent}{0pt}
\setlength{\parskip}{6pt plus 2pt minus 1pt}
}
\setlength{\emergencystretch}{3em}  % prevent overfull lines
\providecommand{\tightlist}{%
  \setlength{\itemsep}{0pt}\setlength{\parskip}{0pt}}
\setcounter{secnumdepth}{0}
% Redefines (sub)paragraphs to behave more like sections
\ifx\paragraph\undefined\else
\let\oldparagraph\paragraph
\renewcommand{\paragraph}[1]{\oldparagraph{#1}\mbox{}}
\fi
\ifx\subparagraph\undefined\else
\let\oldsubparagraph\subparagraph
\renewcommand{\subparagraph}[1]{\oldsubparagraph{#1}\mbox{}}
\fi

%%% Use protect on footnotes to avoid problems with footnotes in titles
\let\rmarkdownfootnote\footnote%
\def\footnote{\protect\rmarkdownfootnote}

%%% Change title format to be more compact
\usepackage{titling}

% Create subtitle command for use in maketitle
\newcommand{\subtitle}[1]{
  \posttitle{
    \begin{center}\large#1\end{center}
    }
}

\setlength{\droptitle}{-2em}

  \title{R\_M\_B}
    \pretitle{\vspace{\droptitle}\centering\huge}
  \posttitle{\par}
    \author{}
    \preauthor{}\postauthor{}
    \date{}
    \predate{}\postdate{}
  

\begin{document}
\maketitle

\begin{Shaded}
\begin{Highlighting}[]
\FunctionTok{clc}\NormalTok{, }\FunctionTok{clear} \FunctionTok{all}\NormalTok{; }\FunctionTok{tic}
\CommentTok{%rozwiązanie analityczne}
\NormalTok{G = @(x,t) }\FunctionTok{sin}\NormalTok{(}\BaseNTok{pi}\NormalTok{.*x./}\FloatTok{2}\NormalTok{).*}\FunctionTok{exp}\NormalTok{(-(}\BaseNTok{pi}\NormalTok{.^}\FloatTok{2}\NormalTok{).*t./}\FloatTok{4}\NormalTok{);}
\CommentTok{%przedział omega}
\NormalTok{xa=}\FloatTok{0}\NormalTok{; xb=}\FloatTok{2}\NormalTok{; yc=}\FloatTok{0}\NormalTok{; yd=}\FloatTok{1}\NormalTok{;}
\CommentTok{%warunki brzegowe}
\NormalTok{u1 = @(x) }\FloatTok{0}\NormalTok{; u2 = @(x) }\FloatTok{0}\NormalTok{; u3 = @(x,t) }\FunctionTok{sin}\NormalTok{(}\BaseNTok{pi}\NormalTok{*x/}\FloatTok{2}\NormalTok{);}
\NormalTok{licznik=}\FloatTok{0}\NormalTok{;}
\CommentTok{%siatka}
\NormalTok{m=}\FloatTok{50}\NormalTok{; D=}\FloatTok{1}\NormalTok{; deltax=(xb-xa)/(m-}\FloatTok{1}\NormalTok{);}
\NormalTok{x=[xa:deltax:xb];         }\CommentTok{%przedział przestrzenny}
\NormalTok{deltat=(deltax^}\FloatTok{2}\NormalTok{)/D; }
\NormalTok{n_end=}\FunctionTok{floor}\NormalTok{(yd/deltat)+}\FloatTok{1}\NormalTok{;}
\NormalTok{t=[}\FloatTok{0}\NormalTok{:deltat:}\FloatTok{1}\NormalTok{];           }\CommentTok{%przedział czasowy}
\NormalTok{theta = }\FloatTok{1}\NormalTok{/}\FloatTok{2}\NormalTok{ - deltax^}\FloatTok{2}\NormalTok{ / (}\FloatTok{12}\NormalTok{ * D * deltat);}
\NormalTok{alfa = D * deltat / deltax^}\FloatTok{2}\NormalTok{;}
\CommentTok{%macierz}
\FunctionTok{psi}\NormalTok{=}\FunctionTok{zeros}\NormalTok{(n_end,}\FunctionTok{length}\NormalTok{(x)); }\CommentTok{%utworzenie pustej macierzy}
\CommentTok{%dodanie warunku początkowego}
\FunctionTok{psi}\NormalTok{(}\FloatTok{1}\NormalTok{,:) = u3(x);}
\FunctionTok{psi}\NormalTok{(:,}\FloatTok{1}\NormalTok{) = u1(t);}
\FunctionTok{psi}\NormalTok{(:,m) = u2(t);}
\NormalTok{A1 = }\FunctionTok{eye}\NormalTok{(m-}\FloatTok{2}\NormalTok{) .* (}\FloatTok{2}\NormalTok{*alfa + }\FloatTok{1}\NormalTok{ + theta);}
\NormalTok{A2 = }\FunctionTok{diag}\NormalTok{(}\FunctionTok{eye}\NormalTok{(m-}\FloatTok{3}\NormalTok{)) * -alfa;}
\NormalTok{A = A1 + }\FunctionTok{diag}\NormalTok{(A2,-}\FloatTok{1}\NormalTok{) + }\FunctionTok{diag}\NormalTok{(A2, }\FloatTok{1}\NormalTok{);}
\CommentTok{% Pierwszy poziom za pomocą schematu dwupoziomowego}
\NormalTok{A_2 = (}\FloatTok{2}\NormalTok{+(deltax^}\FloatTok{2}\NormalTok{)/(deltat))*}\FunctionTok{diag}\NormalTok{(}\FunctionTok{eye}\NormalTok{(m-}\FloatTok{2}\NormalTok{));}
\NormalTok{B_2 = }\FunctionTok{diag}\NormalTok{(A_2) + -}\FloatTok{1}\NormalTok{*}\FunctionTok{diag}\NormalTok{(}\FunctionTok{diag}\NormalTok{(}\FunctionTok{eye}\NormalTok{(m-}\FloatTok{3}\NormalTok{)),-}\FloatTok{1}\NormalTok{) + -}\FloatTok{1}\NormalTok{*}\FunctionTok{diag}\NormalTok{(}\FunctionTok{diag}\NormalTok{(}\FunctionTok{eye}\NormalTok{(m-}\FloatTok{3}\NormalTok{)),}\FloatTok{1}\NormalTok{);}
\NormalTok{F = }\FunctionTok{diag}\NormalTok{(}\FunctionTok{eye}\NormalTok{(m-}\FloatTok{2}\NormalTok{)) * deltax^}\FloatTok{2}\NormalTok{/deltat .* }\FunctionTok{psi}\NormalTok{(}\FloatTok{1}\NormalTok{,}\FloatTok{2}\NormalTok{:m-}\FloatTok{1}\NormalTok{)';}
\NormalTok{F(}\FloatTok{1}\NormalTok{) = F(}\FloatTok{1}\NormalTok{) + }\FunctionTok{psi}\NormalTok{(}\FloatTok{2}\NormalTok{,}\FloatTok{1}\NormalTok{); F(}\FunctionTok{length}\NormalTok{(F)) = F(}\FunctionTok{length}\NormalTok{(F)) + }\FunctionTok{psi}\NormalTok{(}\FloatTok{2}\NormalTok{, m);  }
\FunctionTok{psi}\NormalTok{(}\FloatTok{2}\NormalTok{,}\FloatTok{2}\NormalTok{:m-}\FloatTok{1}\NormalTok{) = linsolve(B_2,F);}
\CommentTok{% Pozostałe poziomy}
\NormalTok{for n=}\FloatTok{3}\NormalTok{:n_end}
\NormalTok{  F = (}\FloatTok{1}\NormalTok{+}\FloatTok{2}\NormalTok{*theta) * }\FunctionTok{psi}\NormalTok{(n-}\FloatTok{1}\NormalTok{,}\FloatTok{2}\NormalTok{:m-}\FloatTok{1}\NormalTok{) - theta * }\FunctionTok{psi}\NormalTok{(n-}\FloatTok{2}\NormalTok{, }\FloatTok{2}\NormalTok{:m-}\FloatTok{1}\NormalTok{);}
\NormalTok{  F(}\FloatTok{1}\NormalTok{) = F(}\FloatTok{1}\NormalTok{) + alfa * }\FunctionTok{psi}\NormalTok{(n, }\FloatTok{1}\NormalTok{);}
\NormalTok{  F(m-}\FloatTok{2}\NormalTok{) = F(m-}\FloatTok{2}\NormalTok{) + alfa * }\FunctionTok{psi}\NormalTok{(n, m);}
  \FunctionTok{psi}\NormalTok{(n,}\FloatTok{2}\NormalTok{:m-}\FloatTok{1}\NormalTok{) = linsolve(A,F');}
\NormalTok{  licznik = licznik+}\FloatTok{1}\NormalTok{;}
\NormalTok{end}
\NormalTok{[X,T] = }\FunctionTok{meshgrid}\NormalTok{(x,t);}
\FunctionTok{subplot}\NormalTok{(}\FloatTok{1}\NormalTok{,}\FloatTok{2}\NormalTok{,}\FloatTok{1}\NormalTok{)}
\FunctionTok{surf}\NormalTok{(X,T,}\FunctionTok{psi}\NormalTok{)}
\FunctionTok{title}\NormalTok{(}\StringTok{'Metoda Numeryczna'}\NormalTok{)}
\FunctionTok{subplot}\NormalTok{(}\FloatTok{1}\NormalTok{,}\FloatTok{2}\NormalTok{,}\FloatTok{2}\NormalTok{)}
\FunctionTok{surf}\NormalTok{(X,T,(G(X,T)))}
\FunctionTok{title}\NormalTok{(}\StringTok{'Metoda Analityczna'}\NormalTok{)}
\NormalTok{Error=}\FunctionTok{max}\NormalTok{(}\FunctionTok{max}\NormalTok{(}\FunctionTok{abs}\NormalTok{(}\FunctionTok{psi}\NormalTok{-G(X,T))));}
\NormalTok{licznik; }\FunctionTok{toc}
\end{Highlighting}
\end{Shaded}

\begin{verbatim}
## Elapsed time is 0.483188 seconds.
\end{verbatim}


\end{document}
