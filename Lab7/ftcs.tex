\begin{Shaded}
\begin{Highlighting}[]
\FunctionTok{clc}\NormalTok{,}\FunctionTok{clear} \FunctionTok{all}
\FunctionTok{tic}
\CommentTok{%rozwiązanie analityczne}
\NormalTok{G = @(x,t) }\FunctionTok{sin}\NormalTok{(}\BaseNTok{pi}\NormalTok{.*x./}\FloatTok{2}\NormalTok{).*}\FunctionTok{exp}\NormalTok{(-(}\BaseNTok{pi}\NormalTok{.^}\FloatTok{2}\NormalTok{).*t./}\FloatTok{4}\NormalTok{);}

\CommentTok{%przedział omega}
\NormalTok{xa=}\FloatTok{0}\NormalTok{;}
\NormalTok{xb=}\FloatTok{2}\NormalTok{;}
\NormalTok{yc=}\FloatTok{0}\NormalTok{;}
\NormalTok{yd=}\FloatTok{1}\NormalTok{;}

\CommentTok{%warunki brzegowe}
\NormalTok{u1 = @(x) }\FloatTok{0}\NormalTok{;}
\NormalTok{u2 = @(x) }\FloatTok{0}\NormalTok{;}
\NormalTok{u3 = @(x,t) }\FunctionTok{sin}\NormalTok{(}\BaseNTok{pi}\NormalTok{*x/}\FloatTok{2}\NormalTok{);}

\NormalTok{licznik=}\FloatTok{0}\NormalTok{;}
\CommentTok{%siatka}
\NormalTok{m=}\FloatTok{30}\NormalTok{;}
\NormalTok{D=}\FloatTok{1}\NormalTok{;}
\NormalTok{deltax=(xb-xa)/(m-}\FloatTok{1}\NormalTok{);}
\NormalTok{x=[xa:deltax:xb];         }\CommentTok{%przedział przestrzenny}
\NormalTok{deltat=(deltax^}\FloatTok{2}\NormalTok{)/(}\FloatTok{20}\NormalTok{*D); }\CommentTok{%dzielimy od razu przez 10, aby wartość }
\NormalTok{n_end=}\FunctionTok{floor}\NormalTok{(yd/deltat)+}\FloatTok{1}\NormalTok{; }\CommentTok{%nie była blisko deltat graniczne}
\NormalTok{t=[}\FloatTok{0}\NormalTok{:deltat:}\FloatTok{1}\NormalTok{];           }\CommentTok{%przedział czasowy}

\CommentTok{%macierz}
\FunctionTok{psi}\NormalTok{=}\FunctionTok{zeros}\NormalTok{(n_end,}\FunctionTok{length}\NormalTok{(x)); }\CommentTok{%utworzenie pustej macierzy}

\NormalTok{for }\BaseNTok{i}\NormalTok{=}\FloatTok{2}\NormalTok{:m-}\FloatTok{1}                \CommentTok{%dodanie warunku początkowego}
  \FunctionTok{psi}\NormalTok{(}\FloatTok{1}\NormalTok{,}\BaseNTok{i}\NormalTok{)=u3(x(}\BaseNTok{i}\NormalTok{));}
\NormalTok{end}
                        
\NormalTok{for n=}\FloatTok{2}\NormalTok{:n_end}
\NormalTok{  for }\BaseNTok{i}\NormalTok{=}\FloatTok{2}\NormalTok{:m-}\FloatTok{1}
    \FunctionTok{psi}\NormalTok{(n,}\BaseNTok{i}\NormalTok{)=(deltat/deltax^}\FloatTok{2}\NormalTok{)*(}\FunctionTok{psi}\NormalTok{(n-}\FloatTok{1}\NormalTok{,}\BaseNTok{i}\NormalTok{+}\FloatTok{1}\NormalTok{)-}\FloatTok{2}\NormalTok{*}\FunctionTok{psi}\NormalTok{(n-}\FloatTok{1}\NormalTok{,}\BaseNTok{i}\NormalTok{)+}\FunctionTok{psi}\NormalTok{(n-}\FloatTok{1}\NormalTok{,}\BaseNTok{i}\NormalTok{-}\FloatTok{1}\NormalTok{))+}\FunctionTok{psi}\NormalTok{(n-}\FloatTok{1}\NormalTok{,}\BaseNTok{i}\NormalTok{);}
\NormalTok{  end}
\NormalTok{  licznik = licznik+}\FloatTok{1}\NormalTok{;}
\NormalTok{end}

\NormalTok{[X,T] = }\FunctionTok{meshgrid}\NormalTok{(x,t);}
\FunctionTok{subplot}\NormalTok{(}\FloatTok{1}\NormalTok{,}\FloatTok{2}\NormalTok{,}\FloatTok{1}\NormalTok{)}
\FunctionTok{surf}\NormalTok{(X,T,}\FunctionTok{psi}\NormalTok{)}
\FunctionTok{title}\NormalTok{(}\StringTok{'Metoda Numeryczna'}\NormalTok{)}
\FunctionTok{subplot}\NormalTok{(}\FloatTok{1}\NormalTok{,}\FloatTok{2}\NormalTok{,}\FloatTok{2}\NormalTok{)}
\FunctionTok{surf}\NormalTok{(X,T,(G(X,T)))}
\FunctionTok{title}\NormalTok{(}\StringTok{'Metoda Analityczna'}\NormalTok{)}
\NormalTok{licznik}
\FunctionTok{toc}
\end{Highlighting}
\end{Shaded}