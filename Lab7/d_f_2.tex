\begin{Shaded}
\begin{Highlighting}[]
\FunctionTok{clc}\NormalTok{,}\FunctionTok{clear} \FunctionTok{all}
\FunctionTok{tic}
\CommentTok{%rozwiązanie analityczne}
\NormalTok{G = @(x,t) }\FunctionTok{sin}\NormalTok{(}\BaseNTok{pi}\NormalTok{.*x./}\FloatTok{2}\NormalTok{).*}\FunctionTok{exp}\NormalTok{(-(}\BaseNTok{pi}\NormalTok{.^}\FloatTok{2}\NormalTok{).*t./}\FloatTok{4}\NormalTok{);}
\CommentTok{%przedział omega}
\NormalTok{xa=}\FloatTok{0}\NormalTok{; xb=}\FloatTok{2}\NormalTok{; yc=}\FloatTok{0}\NormalTok{; yd=}\FloatTok{1}\NormalTok{;}
\CommentTok{%warunki brzegowe}
\NormalTok{u1 = @(x) }\FloatTok{0}\NormalTok{;}
\NormalTok{u2 = @(x) }\FloatTok{0}\NormalTok{;}
\NormalTok{u3 = @(x,t) }\FunctionTok{sin}\NormalTok{(}\BaseNTok{pi}\NormalTok{*x/}\FloatTok{2}\NormalTok{);}
\NormalTok{licznik=}\FloatTok{0}\NormalTok{;}
\CommentTok{%siatka}
\NormalTok{m=}\FloatTok{40}\NormalTok{; D=}\FloatTok{1}\NormalTok{;}
\NormalTok{deltax=(xb-xa)/(m-}\FloatTok{1}\NormalTok{);}
\NormalTok{x=[xa:deltax:xb];         }\CommentTok{%przedział przestrzenny}
\NormalTok{deltat=(deltax^}\FloatTok{2}\NormalTok{)/(}\FloatTok{10}\NormalTok{*D); }
\NormalTok{n_end=}\FunctionTok{floor}\NormalTok{(yd/deltat)+}\FloatTok{1}\NormalTok{;}
\NormalTok{t=[}\FloatTok{0}\NormalTok{:deltat:}\FloatTok{1}\NormalTok{];           }\CommentTok{%przedział czasowy}
\CommentTok{%macierz}
\FunctionTok{psi}\NormalTok{=}\FunctionTok{zeros}\NormalTok{(n_end,}\FunctionTok{length}\NormalTok{(x)); }\CommentTok{%utworzenie pustej macierzy}
\CommentTok{%dodanie warunku początkowego}
\FunctionTok{psi}\NormalTok{(}\FloatTok{1}\NormalTok{,:) = u3(x);}
\FunctionTok{psi}\NormalTok{(:,}\FloatTok{1}\NormalTok{) = u1(t);}
\FunctionTok{psi}\NormalTok{(:,m) = u2(t);}
\CommentTok{% Pierwszy poziom za pomocą schematu dwupoziomowego}
\NormalTok{A_2 = (}\FloatTok{2}\NormalTok{+(deltax^}\FloatTok{2}\NormalTok{)/(deltat))*}\FunctionTok{diag}\NormalTok{(}\FunctionTok{eye}\NormalTok{(m-}\FloatTok{2}\NormalTok{));}
\NormalTok{B_2 = }\FunctionTok{diag}\NormalTok{(A_2) + -}\FloatTok{1}\NormalTok{*}\FunctionTok{diag}\NormalTok{(}\FunctionTok{diag}\NormalTok{(}\FunctionTok{eye}\NormalTok{(m-}\FloatTok{3}\NormalTok{)),-}\FloatTok{1}\NormalTok{) + -}\FloatTok{1}\NormalTok{*}\FunctionTok{diag}\NormalTok{(}\FunctionTok{diag}\NormalTok{(}\FunctionTok{eye}\NormalTok{(m-}\FloatTok{3}\NormalTok{)),}\FloatTok{1}\NormalTok{);}
\NormalTok{F = }\FunctionTok{diag}\NormalTok{(}\FunctionTok{eye}\NormalTok{(m-}\FloatTok{2}\NormalTok{)) * deltax^}\FloatTok{2}\NormalTok{/deltat .* }\FunctionTok{psi}\NormalTok{(}\FloatTok{1}\NormalTok{,}\FloatTok{2}\NormalTok{:m-}\FloatTok{1}\NormalTok{)';}
\NormalTok{F(}\FloatTok{1}\NormalTok{) = F(}\FloatTok{1}\NormalTok{) + }\FunctionTok{psi}\NormalTok{(}\FloatTok{2}\NormalTok{,}\FloatTok{1}\NormalTok{);}
\NormalTok{F(}\FunctionTok{length}\NormalTok{(F)) = F(}\FunctionTok{length}\NormalTok{(F)) + }\FunctionTok{psi}\NormalTok{(}\FloatTok{2}\NormalTok{, m);  }
\FunctionTok{psi}\NormalTok{(}\FloatTok{2}\NormalTok{,}\FloatTok{2}\NormalTok{:m-}\FloatTok{1}\NormalTok{) = linsolve(B_2,F);}
\CommentTok{% Pozostałe poziomy}
\NormalTok{F = }\FunctionTok{diag}\NormalTok{(}\FunctionTok{eye}\NormalTok{(}\FloatTok{2}\NormalTok{*m-}\FloatTok{2}\NormalTok{));}
\NormalTok{A1 = }\FunctionTok{eye}\NormalTok{(m-}\FloatTok{2}\NormalTok{) .* -(D/deltax^}\FloatTok{2}\NormalTok{-}\FloatTok{1}\NormalTok{/(}\FloatTok{2}\NormalTok{*deltat));}
\NormalTok{A2 = (}\FunctionTok{eye}\NormalTok{(m-}\FloatTok{2}\NormalTok{,m) + }\FunctionTok{diag}\NormalTok{(}\FunctionTok{diag}\NormalTok{(}\FunctionTok{eye}\NormalTok{(m-}\FloatTok{2}\NormalTok{)),}\FloatTok{2}\NormalTok{)(}\FloatTok{1}\NormalTok{:m-}\FloatTok{2}\NormalTok{,:)) .* D/deltax^}\FloatTok{2}\NormalTok{;}
\NormalTok{A = [A1, A2] ./ (}\FloatTok{1}\NormalTok{/(}\FloatTok{2}\NormalTok{*deltat) + D/deltax^}\FloatTok{2}\NormalTok{);}
\NormalTok{for n=}\FloatTok{3}\NormalTok{:n_end}
\NormalTok{  F(}\FloatTok{1}\NormalTok{:m-}\FloatTok{2}\NormalTok{) = }\FunctionTok{psi}\NormalTok{(n-}\FloatTok{2}\NormalTok{,}\FloatTok{2}\NormalTok{:m-}\FloatTok{1}\NormalTok{);}
\NormalTok{  F(m-}\FloatTok{1}\NormalTok{:}\FunctionTok{length}\NormalTok{(F)) = }\FunctionTok{psi}\NormalTok{(n-}\FloatTok{1}\NormalTok{,:);}
  \FunctionTok{psi}\NormalTok{(n,}\FloatTok{2}\NormalTok{:m-}\FloatTok{1}\NormalTok{) = A * F;}
\NormalTok{  licznik = licznik+}\FloatTok{1}\NormalTok{;}
\NormalTok{end}
\NormalTok{[X,T] = }\FunctionTok{meshgrid}\NormalTok{(x,t);}
\FunctionTok{subplot}\NormalTok{(}\FloatTok{1}\NormalTok{,}\FloatTok{2}\NormalTok{,}\FloatTok{1}\NormalTok{)}
\FunctionTok{surf}\NormalTok{(X,T,}\FunctionTok{psi}\NormalTok{)}
\FunctionTok{title}\NormalTok{(}\StringTok{'Metoda Numeryczna'}\NormalTok{)}
\FunctionTok{subplot}\NormalTok{(}\FloatTok{1}\NormalTok{,}\FloatTok{2}\NormalTok{,}\FloatTok{2}\NormalTok{)}
\FunctionTok{surf}\NormalTok{(X,T,(G(X,T)))}
\FunctionTok{title}\NormalTok{(}\StringTok{'Metoda Analityczna'}\NormalTok{)}
\NormalTok{Error=}\FunctionTok{max}\NormalTok{(}\FunctionTok{max}\NormalTok{(}\FunctionTok{abs}\NormalTok{(}\FunctionTok{psi}\NormalTok{-G(X,T))));}
\NormalTok{licznik; }\FunctionTok{toc}
\end{Highlighting}
\end{Shaded}