\begin{Shaded}
\begin{Highlighting}[]
\FunctionTok{clc}\NormalTok{,}\FunctionTok{clear} \FunctionTok{all}\NormalTok{; }\FunctionTok{tic}
\CommentTok{%rozwiązanie analityczne}
\NormalTok{G = @(x,t) }\FunctionTok{sin}\NormalTok{(}\BaseNTok{pi}\NormalTok{.*x./}\FloatTok{2}\NormalTok{).*}\FunctionTok{exp}\NormalTok{(-(}\BaseNTok{pi}\NormalTok{.^}\FloatTok{2}\NormalTok{).*t./}\FloatTok{4}\NormalTok{);}
\CommentTok{%przedział omega}
\NormalTok{xa=}\FloatTok{0}\NormalTok{; xb=}\FloatTok{2}\NormalTok{; yc=}\FloatTok{0}\NormalTok{; yd=}\FloatTok{1}\NormalTok{;}
\CommentTok{%warunki brzegowe}
\NormalTok{ub1 = @(t) }\FloatTok{0}\NormalTok{;}
\NormalTok{ub2 = @(t) }\FloatTok{0}\NormalTok{;}
\NormalTok{up = @(x) }\FunctionTok{sin}\NormalTok{(}\BaseNTok{pi}\NormalTok{*x/}\FloatTok{2}\NormalTok{);}
\CommentTok{%siatka}
\NormalTok{n=}\FloatTok{10}\NormalTok{;}
\NormalTok{h=(xb-xa)/(n+}\FloatTok{1}\NormalTok{);}
\NormalTok{k=(yd-yc)/(n+}\FloatTok{1}\NormalTok{);}
\NormalTok{x=xa:h:xb;    }
\NormalTok{t=yc:k:yd;}
\NormalTok{D=}\FloatTok{1}\NormalTok{;}
\NormalTok{alfa = (D*k)/(}\FloatTok{2}\NormalTok{*(h^}\FloatTok{2}\NormalTok{));}
\CommentTok{%tworzenie macierzy A}
\NormalTok{A1 = -alfa*}\FunctionTok{ones}\NormalTok{(n+}\FloatTok{1}\NormalTok{,}\FloatTok{1}\NormalTok{);}
\NormalTok{A2 = (}\FloatTok{2}\NormalTok{*alfa+}\FloatTok{1}\NormalTok{)*}\FunctionTok{ones}\NormalTok{(n+}\FloatTok{2}\NormalTok{,}\FloatTok{1}\NormalTok{);}
\NormalTok{A = }\FunctionTok{diag}\NormalTok{(A1,-}\FloatTok{1}\NormalTok{) + }\FunctionTok{diag}\NormalTok{(A2) + }\FunctionTok{diag}\NormalTok{(A1,}\FloatTok{1}\NormalTok{);}
\NormalTok{A(}\FloatTok{1}\NormalTok{,}\FloatTok{1}\NormalTok{)=}\FloatTok{1}\NormalTok{; A(}\FloatTok{1}\NormalTok{,}\FloatTok{2}\NormalTok{)=}\FloatTok{0}\NormalTok{;}
\NormalTok{A(end,end)=}\FloatTok{1}\NormalTok{; A(end,end-}\FloatTok{1}\NormalTok{)=}\FloatTok{0}\NormalTok{;}
\CommentTok{%utworzenie pustej macierzy}
\NormalTok{U=}\FunctionTok{zeros}\NormalTok{(n+}\FloatTok{2}\NormalTok{,n+}\FloatTok{2}\NormalTok{); }
\CommentTok{%dodanie warunków początkowego/brzegowego}
\NormalTok{U(}\FloatTok{1}\NormalTok{,:) = up(x);}
\NormalTok{U(:,}\FloatTok{1}\NormalTok{) = ub1(t);}
\NormalTok{U(:,n+}\FloatTok{2}\NormalTok{) = ub2(t);}
\FunctionTok{psi}\NormalTok{ = }\FunctionTok{zeros}\NormalTok{(}\FloatTok{1}\NormalTok{,n+}\FloatTok{2}\NormalTok{);}
\NormalTok{licznik=}\FloatTok{0}\NormalTok{;}
\NormalTok{for }\BaseNTok{j}\NormalTok{=}\FloatTok{2}\NormalTok{:n+}\FloatTok{2}
\NormalTok{    for }\BaseNTok{i}\NormalTok{=}\FloatTok{2}\NormalTok{:n+}\FloatTok{1}
        \FunctionTok{psi}\NormalTok{(}\BaseNTok{i}\NormalTok{) = alfa*(U(}\BaseNTok{j}\NormalTok{-}\FloatTok{1}\NormalTok{,}\BaseNTok{i}\NormalTok{+}\FloatTok{1}\NormalTok{)-}\FloatTok{2}\NormalTok{*U(}\BaseNTok{j}\NormalTok{-}\FloatTok{1}\NormalTok{,}\BaseNTok{i}\NormalTok{)+U(}\BaseNTok{j}\NormalTok{-}\FloatTok{1}\NormalTok{,}\BaseNTok{i}\NormalTok{-}\FloatTok{1}\NormalTok{))+U(}\BaseNTok{j}\NormalTok{-}\FloatTok{1}\NormalTok{,}\BaseNTok{i}\NormalTok{);}
\NormalTok{    end}
\NormalTok{    F = linsolve(A,psi');}
\NormalTok{    U(}\BaseNTok{j}\NormalTok{,}\FloatTok{1}\NormalTok{:end)=F;}
\NormalTok{    licznik=licznik+}\FloatTok{1}\NormalTok{;}
\NormalTok{end}
\NormalTok{[X,T] = }\FunctionTok{meshgrid}\NormalTok{(x,t);}
\FunctionTok{subplot}\NormalTok{(}\FloatTok{1}\NormalTok{,}\FloatTok{2}\NormalTok{,}\FloatTok{1}\NormalTok{)}
\FunctionTok{surf}\NormalTok{(X,T,U)}
\FunctionTok{title}\NormalTok{(}\StringTok{'Metoda Numeryczna'}\NormalTok{)}
\FunctionTok{subplot}\NormalTok{(}\FloatTok{1}\NormalTok{,}\FloatTok{2}\NormalTok{,}\FloatTok{2}\NormalTok{)}
\FunctionTok{surf}\NormalTok{(X,T,(G(X,T)))}
\FunctionTok{title}\NormalTok{(}\StringTok{'Metoda Analityczna'}\NormalTok{)}
\NormalTok{Error=}\FunctionTok{max}\NormalTok{(}\FunctionTok{max}\NormalTok{(}\FunctionTok{abs}\NormalTok{(U-G(X,T))))}
\NormalTok{licznik; }\FunctionTok{toc}
\end{Highlighting}
\end{Shaded}